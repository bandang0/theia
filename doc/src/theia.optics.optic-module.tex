%
% API Documentation for theia
% Module theia.optics.optic
%
% Generated by epydoc 3.0.1
% [Tue Jun  6 14:55:52 2017]
%

%%%%%%%%%%%%%%%%%%%%%%%%%%%%%%%%%%%%%%%%%%%%%%%%%%%%%%%%%%%%%%%%%%%%%%%%%%%
%%                          Module Description                           %%
%%%%%%%%%%%%%%%%%%%%%%%%%%%%%%%%%%%%%%%%%%%%%%%%%%%%%%%%%%%%%%%%%%%%%%%%%%%

    \index{theia \textit{(package)}!theia.optics \textit{(package)}!theia.optics.optic \textit{(module)}|(}
\section{Module theia.optics.optic}

    \label{theia:optics:optic}
Defines the Optic class for theia.


%%%%%%%%%%%%%%%%%%%%%%%%%%%%%%%%%%%%%%%%%%%%%%%%%%%%%%%%%%%%%%%%%%%%%%%%%%%
%%                               Variables                               %%
%%%%%%%%%%%%%%%%%%%%%%%%%%%%%%%%%%%%%%%%%%%%%%%%%%%%%%%%%%%%%%%%%%%%%%%%%%%

  \subsection{Variables}

    \vspace{-1cm}
\hspace{\varindent}\begin{longtable}{|p{\varnamewidth}|p{\vardescrwidth}|l}
\cline{1-2}
\cline{1-2} \centering \textbf{Name} & \centering \textbf{Description}& \\
\cline{1-2}
\endhead\cline{1-2}\multicolumn{3}{r}{\small\textit{continued on next page}}\\\endfoot\cline{1-2}
\endlastfoot\raggedright \_\-\_\-p\-a\-c\-k\-a\-g\-e\-\_\-\_\- & \raggedright \textbf{Value:} 
{\tt \texttt{'}\texttt{theia.optics}\texttt{'}}&\\
\cline{1-2}
\end{longtable}


%%%%%%%%%%%%%%%%%%%%%%%%%%%%%%%%%%%%%%%%%%%%%%%%%%%%%%%%%%%%%%%%%%%%%%%%%%%
%%                           Class Description                           %%
%%%%%%%%%%%%%%%%%%%%%%%%%%%%%%%%%%%%%%%%%%%%%%%%%%%%%%%%%%%%%%%%%%%%%%%%%%%

    \index{theia \textit{(package)}!theia.optics \textit{(package)}!theia.optics.optic \textit{(module)}!theia.optics.optic.Optic \textit{(class)}|(}
\subsection{Class Optic}

    \label{theia:optics:optic:Optic}
\begin{tabular}{cccccccc}
% Line for object, linespec=[False, False]
\multicolumn{2}{r}{\settowidth{\BCL}{object}\multirow{2}{\BCL}{object}}
&&
&&
  \\\cline{3-3}
  &&\multicolumn{1}{c|}{}
&&
&&
  \\
% Line for theia.optics.component.SetupComponent, linespec=[False]
\multicolumn{4}{r}{\settowidth{\BCL}{theia.optics.component.SetupComponent}\multirow{2}{\BCL}{theia.optics.component.SetupComponent}}
&&
  \\\cline{5-5}
  &&&&\multicolumn{1}{c|}{}
&&
  \\
&&&&\multicolumn{2}{l}{\textbf{theia.optics.optic.Optic}}
\end{tabular}

\textbf{Known Subclasses:}
theia.optics.lens.Lens,
    theia.optics.mirror.Mirror

\begin{alltt}


Optic class.

This class is a base class for optics which may interact with Gaussian
beams and return transmitted and reflected beams (mirrors, lenses, etc.)


*=== Attributes ===*
SetupCount (inherited): class attribute, counts all setup components.
    [integer]
OptCount: class attribute, counts optical components. [string]
HRCenter (inherited): center of the 'chord' of the HR surface. [3D vector]
HRNorm (inherited): unitary normal to the 'chord' of the HR (always pointing
 towards the outside of the component). [3D vector]
Thick (inherited): thickness of the optic, counted in opposite direction to
    HRNorm. [float]
Dia (inherited): diameter of the component. [float]
Name (inherited): name of the component. [string]
Ref (inherited): reference string (for keeping track with the lab). [string]
ARCenter: center of the 'chord' of the AR surface. [3D vector]
ARNorm: unitary normal to the 'chord' of the AR (always pointing
 towards the outside of the component). [3D vector]
N: refraction index of the material. [float]
HRK, ARK: curvature of the HR, AR surfaces. [float]
HRr, HRt, ARr, ARt: power reflectance and transmission coefficients of
    the HR and AR surfaces. [float]
KeepI: whether of not to keep data of rays for interference calculations
        on the HR. [boolean]

**Note**: the curvature of any surface is positive for a concave surface
(coating inside the sphere).
Thus kurv*HRNorm/{\textbar}kurv{\textbar} always points to the center
of the sphere of the surface, as is the convention for the lineSurfInter of
geometry module. Same for AR.

*******     HRK {\textgreater} 0 and ARK {\textgreater} 0     *******           HRK {\textgreater} 0 and ARK {\textless} 0
 *****                               ********         and {\textbar}ARK{\textbar} {\textgreater} {\textbar}HRK{\textbar}
 H***A                               H*********A
 *****                               ********
*******                             *******
\end{alltt}


%%%%%%%%%%%%%%%%%%%%%%%%%%%%%%%%%%%%%%%%%%%%%%%%%%%%%%%%%%%%%%%%%%%%%%%%%%%
%%                                Methods                                %%
%%%%%%%%%%%%%%%%%%%%%%%%%%%%%%%%%%%%%%%%%%%%%%%%%%%%%%%%%%%%%%%%%%%%%%%%%%%

  \subsubsection{Methods}

    \vspace{0.5ex}

\hspace{.8\funcindent}\begin{boxedminipage}{\funcwidth}

    \raggedright \textbf{\_\_init\_\_}(\textit{self}, \textit{ARCenter}, \textit{ARNorm}, \textit{N}, \textit{HRK}, \textit{ARK}, \textit{ARr}, \textit{ARt}, \textit{HRr}, \textit{HRt}, \textit{KeepI}, \textit{HRCenter}, \textit{HRNorm}, \textit{Thickness}, \textit{Diameter}, \textit{Name}, \textit{Ref})

    \vspace{-1.5ex}

    \rule{\textwidth}{0.5\fboxrule}
\setlength{\parskip}{2ex}
    Optic base initializer.

    Parameters are the attributes of the object to construct.

    Returns an Optic.

\setlength{\parskip}{1ex}
      Overrides: object.\_\_init\_\_

    \end{boxedminipage}

    \label{theia:optics:optic:Optic:collision}
    \index{theia \textit{(package)}!theia.optics \textit{(package)}!theia.optics.optic \textit{(module)}!theia.optics.optic.Optic \textit{(class)}!theia.optics.optic.Optic.collision \textit{(method)}}

    \vspace{0.5ex}

\hspace{.8\funcindent}\begin{boxedminipage}{\funcwidth}

    \raggedright \textbf{collision}(\textit{self})

    \vspace{-1.5ex}

    \rule{\textwidth}{0.5\fboxrule}
\setlength{\parskip}{2ex}
    Determine whether the HR and AR surfaces intersect.

    Returns True if there is an intersection, False if not.

\setlength{\parskip}{1ex}
    \end{boxedminipage}

    \label{theia:optics:optic:Optic:geoCheck}
    \index{theia \textit{(package)}!theia.optics \textit{(package)}!theia.optics.optic \textit{(module)}!theia.optics.optic.Optic \textit{(class)}!theia.optics.optic.Optic.geoCheck \textit{(method)}}

    \vspace{0.5ex}

\hspace{.8\funcindent}\begin{boxedminipage}{\funcwidth}

    \raggedright \textbf{geoCheck}(\textit{self}, \textit{word})

    \vspace{-1.5ex}

    \rule{\textwidth}{0.5\fboxrule}
\setlength{\parskip}{2ex}
    Makes geometrical checks on surfaces and warns when necessary.

\setlength{\parskip}{1ex}
    \end{boxedminipage}

    \label{theia:optics:optic:Optic:hitSide}
    \index{theia \textit{(package)}!theia.optics \textit{(package)}!theia.optics.optic \textit{(module)}!theia.optics.optic.Optic \textit{(class)}!theia.optics.optic.Optic.hitSide \textit{(method)}}

    \vspace{0.5ex}

\hspace{.8\funcindent}\begin{boxedminipage}{\funcwidth}

    \raggedright \textbf{hitSide}(\textit{self}, \textit{beam})

    \vspace{-1.5ex}

    \rule{\textwidth}{0.5\fboxrule}
\setlength{\parskip}{2ex}
    Compute the daughter beams after interaction on Side at point.

    Generic function: all sides stop beams.

    beam: incident beam. [GaussianBeam]

    Returns \{'t': None, 'r': None\}

\setlength{\parskip}{1ex}
    \end{boxedminipage}


\large{\textbf{\textit{Inherited from theia.optics.component.SetupComponent\textit{(Section \ref{theia:optics:component:SetupComponent})}}}}

\begin{quote}
\_\_str\_\_(), isHit(), lines()
\end{quote}

\large{\textbf{\textit{Inherited from object}}}

\begin{quote}
\_\_delattr\_\_(), \_\_format\_\_(), \_\_getattribute\_\_(), \_\_hash\_\_(), \_\_new\_\_(), \_\_reduce\_\_(), \_\_reduce\_ex\_\_(), \_\_repr\_\_(), \_\_setattr\_\_(), \_\_sizeof\_\_(), \_\_subclasshook\_\_()
\end{quote}

%%%%%%%%%%%%%%%%%%%%%%%%%%%%%%%%%%%%%%%%%%%%%%%%%%%%%%%%%%%%%%%%%%%%%%%%%%%
%%                              Properties                               %%
%%%%%%%%%%%%%%%%%%%%%%%%%%%%%%%%%%%%%%%%%%%%%%%%%%%%%%%%%%%%%%%%%%%%%%%%%%%

  \subsubsection{Properties}

    \vspace{-1cm}
\hspace{\varindent}\begin{longtable}{|p{\varnamewidth}|p{\vardescrwidth}|l}
\cline{1-2}
\cline{1-2} \centering \textbf{Name} & \centering \textbf{Description}& \\
\cline{1-2}
\endhead\cline{1-2}\multicolumn{3}{r}{\small\textit{continued on next page}}\\\endfoot\cline{1-2}
\endlastfoot\multicolumn{2}{|l|}{\textit{Inherited from object}}\\
\multicolumn{2}{|p{\varwidth}|}{\raggedright \_\_class\_\_}\\
\cline{1-2}
\end{longtable}


%%%%%%%%%%%%%%%%%%%%%%%%%%%%%%%%%%%%%%%%%%%%%%%%%%%%%%%%%%%%%%%%%%%%%%%%%%%
%%                            Class Variables                            %%
%%%%%%%%%%%%%%%%%%%%%%%%%%%%%%%%%%%%%%%%%%%%%%%%%%%%%%%%%%%%%%%%%%%%%%%%%%%

  \subsubsection{Class Variables}

    \vspace{-1cm}
\hspace{\varindent}\begin{longtable}{|p{\varnamewidth}|p{\vardescrwidth}|l}
\cline{1-2}
\cline{1-2} \centering \textbf{Name} & \centering \textbf{Description}& \\
\cline{1-2}
\endhead\cline{1-2}\multicolumn{3}{r}{\small\textit{continued on next page}}\\\endfoot\cline{1-2}
\endlastfoot\raggedright O\-p\-t\-C\-o\-u\-n\-t\- & \raggedright \textbf{Value:} 
{\tt 0}&\\
\cline{1-2}
\multicolumn{2}{|l|}{\textit{Inherited from theia.optics.component.SetupComponent \textit{(Section \ref{theia:optics:component:SetupComponent})}}}\\
\multicolumn{2}{|p{\varwidth}|}{\raggedright SetupCount, \_\_abstractmethods\_\_}\\
\cline{1-2}
\end{longtable}

    \index{theia \textit{(package)}!theia.optics \textit{(package)}!theia.optics.optic \textit{(module)}!theia.optics.optic.Optic \textit{(class)}|)}
    \index{theia \textit{(package)}!theia.optics \textit{(package)}!theia.optics.optic \textit{(module)}|)}
