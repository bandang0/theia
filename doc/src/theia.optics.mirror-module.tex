%
% API Documentation for theia
% Module theia.optics.mirror
%
% Generated by epydoc 3.0.1
% [Mon Jul 24 12:23:27 2017]
%

%%%%%%%%%%%%%%%%%%%%%%%%%%%%%%%%%%%%%%%%%%%%%%%%%%%%%%%%%%%%%%%%%%%%%%%%%%%
%%                          Module Description                           %%
%%%%%%%%%%%%%%%%%%%%%%%%%%%%%%%%%%%%%%%%%%%%%%%%%%%%%%%%%%%%%%%%%%%%%%%%%%%

    \index{theia \textit{(package)}!theia.optics \textit{(package)}!theia.optics.mirror \textit{(module)}|(}
\section{Module theia.optics.mirror}

    \label{theia:optics:mirror}
Defines the Mirror class for theia.


%%%%%%%%%%%%%%%%%%%%%%%%%%%%%%%%%%%%%%%%%%%%%%%%%%%%%%%%%%%%%%%%%%%%%%%%%%%
%%                               Variables                               %%
%%%%%%%%%%%%%%%%%%%%%%%%%%%%%%%%%%%%%%%%%%%%%%%%%%%%%%%%%%%%%%%%%%%%%%%%%%%

  \subsection{Variables}

    \vspace{-1cm}
\hspace{\varindent}\begin{longtable}{|p{\varnamewidth}|p{\vardescrwidth}|l}
\cline{1-2}
\cline{1-2} \centering \textbf{Name} & \centering \textbf{Description}& \\
\cline{1-2}
\endhead\cline{1-2}\multicolumn{3}{r}{\small\textit{continued on next page}}\\\endfoot\cline{1-2}
\endlastfoot\raggedright \_\-\_\-p\-a\-c\-k\-a\-g\-e\-\_\-\_\- & \raggedright \textbf{Value:} 
{\tt \texttt{'}\texttt{theia.optics}\texttt{'}}&\\
\cline{1-2}
\end{longtable}


%%%%%%%%%%%%%%%%%%%%%%%%%%%%%%%%%%%%%%%%%%%%%%%%%%%%%%%%%%%%%%%%%%%%%%%%%%%
%%                           Class Description                           %%
%%%%%%%%%%%%%%%%%%%%%%%%%%%%%%%%%%%%%%%%%%%%%%%%%%%%%%%%%%%%%%%%%%%%%%%%%%%

    \index{theia \textit{(package)}!theia.optics \textit{(package)}!theia.optics.mirror \textit{(module)}!theia.optics.mirror.Mirror \textit{(class)}|(}
\subsection{Class Mirror}

    \label{theia:optics:mirror:Mirror}
\begin{tabular}{cccccccccc}
% Line for object, linespec=[False, False, False]
\multicolumn{2}{r}{\settowidth{\BCL}{object}\multirow{2}{\BCL}{object}}
&&
&&
&&
  \\\cline{3-3}
  &&\multicolumn{1}{c|}{}
&&
&&
&&
  \\
% Line for theia.optics.component.SetupComponent, linespec=[False, False]
\multicolumn{4}{r}{\settowidth{\BCL}{theia.optics.component.SetupComponent}\multirow{2}{\BCL}{theia.optics.component.SetupComponent}}
&&
&&
  \\\cline{5-5}
  &&&&\multicolumn{1}{c|}{}
&&
&&
  \\
% Line for theia.optics.optic.Optic, linespec=[False]
\multicolumn{6}{r}{\settowidth{\BCL}{theia.optics.optic.Optic}\multirow{2}{\BCL}{theia.optics.optic.Optic}}
&&
  \\\cline{7-7}
  &&&&&&\multicolumn{1}{c|}{}
&&
  \\
&&&&&&\multicolumn{2}{l}{\textbf{theia.optics.mirror.Mirror}}
\end{tabular}

\begin{alltt}


Mirror class.

This class represents semi reflective mirrors composed of two faces (HR, AR)
and with a wedge angle. These are the objects with which the beams will
interqct during the ray tracing. Please see the documentation for details
on the geometric construction of these mirrors.

*=== Attributes ===*
SetupCount (inherited): class attribute, counts all setup components.
    [integer]
OptCount (inherited): class attribute, counts optical components. [integer]
Name: class attribute. [string]
HRCenter (inherited): center of the 'chord' of the HR surface. [3D vector]
HRNorm (inherited): unitary normal to the 'chord' of the HR (always pointing
 towards the outside of the component). [3D vector]
Thick (inherited): thickness of the optic, counted in opposite direction to
    HRNorm. [float]
Dia (inherited): diameter of the component. [float]
Ref (inherited): reference string (for keeping track with the lab). [string]
ARCenter (inherited): center of the 'chord' of the AR surface. [3D vector]
ARNorm (inherited): unitary normal to the 'chord' of the AR (always pointing
    towards the outside of the component). [3D vector]
N (inherited): refraction index of the material. [float]
HRK, ARK (inherited): curvature of the HR, AR surfaces. [float]
HRr, HRt, ARr, ARt (inherited): power reflectance and transmission
coefficients of the HR and AR surfaces. [float]
KeepI (inherited): whether of not to keep data of rays for interference
calculations on the HR. [boolean]
Wedge: wedge angle of the mirror, please refer to the documentation for
    detaild on the geometry of mirrors and their implementation here.
    [float]
Alpha: rotation alngle used in the geometrical construction of the mirror
    (see doc, it is the amgle between the projection of Ex on the AR plane
    and the vector from ARCenter to the point where the cylinder and the AR
    face meet). [float]

**Note**: the curvature of any surface is positive for a concave surface
(coating inside the sphere).
Thus kurv*HRNorm/{\textbar}kurv{\textbar} always points to the center
of the sphere of the surface, as is the convention for the lineSurfInter of
geometry module. Same for AR.

*******     HRK {\textgreater} 0 and ARK {\textgreater} 0     *******           HRK {\textgreater} 0 and ARK {\textless} 0
 *****                               ********         and {\textbar}ARK{\textbar} {\textgreater} {\textbar}HRK{\textbar}
 H***A                               H*********A
 *****                               ********
*******                             *******
\end{alltt}


%%%%%%%%%%%%%%%%%%%%%%%%%%%%%%%%%%%%%%%%%%%%%%%%%%%%%%%%%%%%%%%%%%%%%%%%%%%
%%                                Methods                                %%
%%%%%%%%%%%%%%%%%%%%%%%%%%%%%%%%%%%%%%%%%%%%%%%%%%%%%%%%%%%%%%%%%%%%%%%%%%%

  \subsubsection{Methods}

    \vspace{0.5ex}

\hspace{.8\funcindent}\begin{boxedminipage}{\funcwidth}

    \raggedright \textbf{\_\_init\_\_}(\textit{self}, \textit{Wedge}={\tt 0.0}, \textit{Alpha}={\tt 0.0}, \textit{X}={\tt 0.0}, \textit{Y}={\tt 0.0}, \textit{Z}={\tt 0.0}, \textit{Theta}={\tt 1.57079632679}, \textit{Phi}={\tt 0.0}, \textit{Diameter}={\tt 0.1}, \textit{HRr}={\tt 0.99}, \textit{HRt}={\tt 0.01}, \textit{ARr}={\tt 0.1}, \textit{ARt}={\tt 0.9}, \textit{HRK}={\tt 0.01}, \textit{ARK}={\tt 0}, \textit{Thickness}={\tt 0.02}, \textit{N}={\tt 1.4585}, \textit{KeepI}={\tt False}, \textit{Ref}={\tt None})

    \vspace{-1.5ex}

    \rule{\textwidth}{0.5\fboxrule}
\setlength{\parskip}{2ex}
    Mirror initializer.

    Parameters are the attributes and the angles theta and phi are 
    spherical coordinates of HRNorm.

    Returns a mirror.

\setlength{\parskip}{1ex}
      Overrides: object.\_\_init\_\_

    \end{boxedminipage}

    \vspace{0.5ex}

\hspace{.8\funcindent}\begin{boxedminipage}{\funcwidth}

    \raggedright \textbf{lines}(\textit{self})

    \vspace{-1.5ex}

    \rule{\textwidth}{0.5\fboxrule}
\setlength{\parskip}{2ex}
    Returns the list of lines necessary to print the object.

\setlength{\parskip}{1ex}
      Overrides: theia.optics.component.SetupComponent.lines

    \end{boxedminipage}

    \vspace{0.5ex}

\hspace{.8\funcindent}\begin{boxedminipage}{\funcwidth}

    \raggedright \textbf{isHit}(\textit{self}, \textit{beam})

    \vspace{-1.5ex}

    \rule{\textwidth}{0.5\fboxrule}
\setlength{\parskip}{2ex}
\begin{alltt}
Determine if a beam hits the Optic.

This is a function for mirrors, using their geometrical
attributes. This uses the line***Inter functions from the geometry
module to find characteristics of impact of beams on mirrors.

beam: incoming beam. [GaussianBeam]

Returns a dictionary with keys:
    'isHit': whether the beam hits the optic. [boolean]
    'intersection point': point in space where it is first hit.
            [3D vector]
    'face': to indicate which face is first hit, can be 'HR', 'AR' or
        'Side'. [string]
    'distance': geometrical distance from beam origin to impact. [float]
\end{alltt}

\setlength{\parskip}{1ex}
      Overrides: theia.optics.component.SetupComponent.isHit

    \end{boxedminipage}

    \vspace{0.5ex}

\hspace{.8\funcindent}\begin{boxedminipage}{\funcwidth}

    \raggedright \textbf{hit}(\textit{self}, \textit{beam}, \textit{order}, \textit{threshold})

    \vspace{-1.5ex}

    \rule{\textwidth}{0.5\fboxrule}
\setlength{\parskip}{2ex}
\begin{alltt}
Compute the refracted and reflected beams after interaction.

The beams returned are those selected after the order and threshold
criterion.

beam: incident beam. [GaussianBeam]
order: maximum strayness of daughter beams, whixh are not returned if
    their strayness is over this order. [integer]
threshold: idem for the power of the daughter beams. [float]

Returns a dictionary of beams with keys:
    't': refracted beam. [GaussianBeam]
    'r': reflected beam. [GaussianBeam]
\end{alltt}

\setlength{\parskip}{1ex}
      Overrides: theia.optics.component.SetupComponent.hit

    \end{boxedminipage}

    \label{theia:optics:mirror:Mirror:hitHR}
    \index{theia \textit{(package)}!theia.optics \textit{(package)}!theia.optics.mirror \textit{(module)}!theia.optics.mirror.Mirror \textit{(class)}!theia.optics.mirror.Mirror.hitHR \textit{(method)}}

    \vspace{0.5ex}

\hspace{.8\funcindent}\begin{boxedminipage}{\funcwidth}

    \raggedright \textbf{hitHR}(\textit{self}, \textit{beam}, \textit{point}, \textit{order}, \textit{threshold})

    \vspace{-1.5ex}

    \rule{\textwidth}{0.5\fboxrule}
\setlength{\parskip}{2ex}
\begin{alltt}
Compute the daughter beams after interaction on HR at point.

beam: incident beam. [GaussianBeam]
point: point in space of interaction. [3D vector]
order: maximum strayness of daughter beams, whixh are not returned if
    their strayness is over this order. [integer]
threshold: idem for the power of the daughter beams. [float]

Returns a dictionary of beams with keys:
    't': refracted beam. [GaussianBeam]
    'r': reflected beam. [GaussianBeam]
\end{alltt}

\setlength{\parskip}{1ex}
    \end{boxedminipage}

    \label{theia:optics:mirror:Mirror:hitAR}
    \index{theia \textit{(package)}!theia.optics \textit{(package)}!theia.optics.mirror \textit{(module)}!theia.optics.mirror.Mirror \textit{(class)}!theia.optics.mirror.Mirror.hitAR \textit{(method)}}

    \vspace{0.5ex}

\hspace{.8\funcindent}\begin{boxedminipage}{\funcwidth}

    \raggedright \textbf{hitAR}(\textit{self}, \textit{beam}, \textit{point}, \textit{order}, \textit{threshold})

    \vspace{-1.5ex}

    \rule{\textwidth}{0.5\fboxrule}
\setlength{\parskip}{2ex}
\begin{alltt}
Compute the daughter beams after interaction on AR at point.

beam: incident beam. [GaussianBeam]
point: point in space of interaction. [3D vector]
order: maximum strayness of daughter beams, which are not returned if
    their strayness is over this order. [integer]
threshold: idem for the power of the daughter beams. [float]

Returns a dictionary of beams with keys:
    't': refracted beam. [GaussianBeam]
    'r': reflected beam. [GaussianBeam]
\end{alltt}

\setlength{\parskip}{1ex}
    \end{boxedminipage}


\large{\textbf{\textit{Inherited from theia.optics.optic.Optic\textit{(Section \ref{theia:optics:optic:Optic})}}}}

\begin{quote}
apexes(), collision(), geoCheck(), hitSide(), translate()
\end{quote}

\large{\textbf{\textit{Inherited from theia.optics.component.SetupComponent\textit{(Section \ref{theia:optics:component:SetupComponent})}}}}

\begin{quote}
\_\_str\_\_()
\end{quote}

\large{\textbf{\textit{Inherited from object}}}

\begin{quote}
\_\_delattr\_\_(), \_\_format\_\_(), \_\_getattribute\_\_(), \_\_hash\_\_(), \_\_new\_\_(), \_\_reduce\_\_(), \_\_reduce\_ex\_\_(), \_\_repr\_\_(), \_\_setattr\_\_(), \_\_sizeof\_\_(), \_\_subclasshook\_\_()
\end{quote}

%%%%%%%%%%%%%%%%%%%%%%%%%%%%%%%%%%%%%%%%%%%%%%%%%%%%%%%%%%%%%%%%%%%%%%%%%%%
%%                              Properties                               %%
%%%%%%%%%%%%%%%%%%%%%%%%%%%%%%%%%%%%%%%%%%%%%%%%%%%%%%%%%%%%%%%%%%%%%%%%%%%

  \subsubsection{Properties}

    \vspace{-1cm}
\hspace{\varindent}\begin{longtable}{|p{\varnamewidth}|p{\vardescrwidth}|l}
\cline{1-2}
\cline{1-2} \centering \textbf{Name} & \centering \textbf{Description}& \\
\cline{1-2}
\endhead\cline{1-2}\multicolumn{3}{r}{\small\textit{continued on next page}}\\\endfoot\cline{1-2}
\endlastfoot\multicolumn{2}{|l|}{\textit{Inherited from object}}\\
\multicolumn{2}{|p{\varwidth}|}{\raggedright \_\_class\_\_}\\
\cline{1-2}
\end{longtable}


%%%%%%%%%%%%%%%%%%%%%%%%%%%%%%%%%%%%%%%%%%%%%%%%%%%%%%%%%%%%%%%%%%%%%%%%%%%
%%                            Class Variables                            %%
%%%%%%%%%%%%%%%%%%%%%%%%%%%%%%%%%%%%%%%%%%%%%%%%%%%%%%%%%%%%%%%%%%%%%%%%%%%

  \subsubsection{Class Variables}

    \vspace{-1cm}
\hspace{\varindent}\begin{longtable}{|p{\varnamewidth}|p{\vardescrwidth}|l}
\cline{1-2}
\cline{1-2} \centering \textbf{Name} & \centering \textbf{Description}& \\
\cline{1-2}
\endhead\cline{1-2}\multicolumn{3}{r}{\small\textit{continued on next page}}\\\endfoot\cline{1-2}
\endlastfoot\raggedright N\-a\-m\-e\- & \raggedright \textbf{Value:} 
{\tt \texttt{'}\texttt{Mirror}\texttt{'}}&\\
\cline{1-2}
\raggedright \_\-\_\-a\-b\-s\-t\-r\-a\-c\-t\-m\-e\-t\-h\-o\-d\-s\-\_\-\_\- & \raggedright \textbf{Value:} 
{\tt \texttt{frozenset([}\texttt{])}}&\\
\cline{1-2}
\multicolumn{2}{|l|}{\textit{Inherited from theia.optics.optic.Optic \textit{(Section \ref{theia:optics:optic:Optic})}}}\\
\multicolumn{2}{|p{\varwidth}|}{\raggedright OptCount}\\
\cline{1-2}
\multicolumn{2}{|l|}{\textit{Inherited from theia.optics.component.SetupComponent \textit{(Section \ref{theia:optics:component:SetupComponent})}}}\\
\multicolumn{2}{|p{\varwidth}|}{\raggedright SetupCount}\\
\cline{1-2}
\end{longtable}

    \index{theia \textit{(package)}!theia.optics \textit{(package)}!theia.optics.mirror \textit{(module)}!theia.optics.mirror.Mirror \textit{(class)}|)}
    \index{theia \textit{(package)}!theia.optics \textit{(package)}!theia.optics.mirror \textit{(module)}|)}
