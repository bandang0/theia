%
% API Documentation for theia
% Module theia.optics.beamdump
%
% Generated by epydoc 3.0.1
% [Fri Aug 18 08:24:16 2017]
%

%%%%%%%%%%%%%%%%%%%%%%%%%%%%%%%%%%%%%%%%%%%%%%%%%%%%%%%%%%%%%%%%%%%%%%%%%%%
%%                          Module Description                           %%
%%%%%%%%%%%%%%%%%%%%%%%%%%%%%%%%%%%%%%%%%%%%%%%%%%%%%%%%%%%%%%%%%%%%%%%%%%%

    \index{theia \textit{(package)}!theia.optics \textit{(package)}!theia.optics.beamdump \textit{(module)}|(}
\section{Module theia.optics.beamdump}

    \label{theia:optics:beamdump}
Defines the BeamDump class for theia.


%%%%%%%%%%%%%%%%%%%%%%%%%%%%%%%%%%%%%%%%%%%%%%%%%%%%%%%%%%%%%%%%%%%%%%%%%%%
%%                           Class Description                           %%
%%%%%%%%%%%%%%%%%%%%%%%%%%%%%%%%%%%%%%%%%%%%%%%%%%%%%%%%%%%%%%%%%%%%%%%%%%%

    \index{theia \textit{(package)}!theia.optics \textit{(package)}!theia.optics.beamdump \textit{(module)}!theia.optics.beamdump.BeamDump \textit{(class)}|(}
\subsection{Class BeamDump}

    \label{theia:optics:beamdump:BeamDump}
\begin{tabular}{cccccccc}
% Line for object, linespec=[False, False]
\multicolumn{2}{r}{\settowidth{\BCL}{object}\multirow{2}{\BCL}{object}}
&&
&&
  \\\cline{3-3}
  &&\multicolumn{1}{c|}{}
&&
&&
  \\
% Line for theia.optics.component.SetupComponent, linespec=[False]
\multicolumn{4}{r}{\settowidth{\BCL}{theia.optics.component.SetupComponent}\multirow{2}{\BCL}{theia.optics.component.SetupComponent}}
&&
  \\\cline{5-5}
  &&&&\multicolumn{1}{c|}{}
&&
  \\
&&&&\multicolumn{2}{l}{\textbf{theia.optics.beamdump.BeamDump}}
\end{tabular}

\begin{alltt}


BeamDump class.

This class represents components on which rays stop. They have cylindrical
symmetry and stop beams on all their faces. They can represent baffles
for example.

*=== Attributes ===*
SetupCount (inherited): class attribute, counts all setup components.
    [integer]
Name: class attribute. [string]
HRCenter (inherited): center of the principal face of the BeamDump in space.
    [3D vector]
ARCenter (inherited): center of the secondary face of the BeamDump in space.
    [3D vector]
HRnorm (inherited): normal unitary vector the this principal face,
    supposed to point outside the media. [3D vector]
Thick (inherited): thickness of the dump, counted in opposite direction to
    HRNorm. [float]
Dia (inherited): diameter of the component. [float]
Ref (inherited): reference string (for keeping track with the lab). [string]
\end{alltt}


%%%%%%%%%%%%%%%%%%%%%%%%%%%%%%%%%%%%%%%%%%%%%%%%%%%%%%%%%%%%%%%%%%%%%%%%%%%
%%                                Methods                                %%
%%%%%%%%%%%%%%%%%%%%%%%%%%%%%%%%%%%%%%%%%%%%%%%%%%%%%%%%%%%%%%%%%%%%%%%%%%%

  \subsubsection{Methods}

    \vspace{0.5ex}

\hspace{.8\funcindent}\begin{boxedminipage}{\funcwidth}

    \raggedright \textbf{\_\_init\_\_}(\textit{self}, \textit{X}={\tt 0.}, \textit{Y}={\tt 0.}, \textit{Z}={\tt 0.}, \textit{Theta}={\tt pi/2.}, \textit{Phi}={\tt 0.}, \textit{Ref}={\tt None}, \textit{Thickness}={\tt 2.e-2}, \textit{Diameter}={\tt 5.e-2})

    \vspace{-1.5ex}

    \rule{\textwidth}{0.5\fboxrule}
\setlength{\parskip}{2ex}
    BeamDump initializer.

    Parameters are the attributes.

    Returns a BeamDump.

\setlength{\parskip}{1ex}
      Overrides: object.\_\_init\_\_

    \end{boxedminipage}

    \vspace{0.5ex}

\hspace{.8\funcindent}\begin{boxedminipage}{\funcwidth}

    \raggedright \textbf{lines}(\textit{self})

    \vspace{-1.5ex}

    \rule{\textwidth}{0.5\fboxrule}
\setlength{\parskip}{2ex}
    Return the list of lines needed to print the object.

\setlength{\parskip}{1ex}
      Overrides: theia.optics.component.SetupComponent.lines

    \end{boxedminipage}

    \vspace{0.5ex}

\hspace{.8\funcindent}\begin{boxedminipage}{\funcwidth}

    \raggedright \textbf{isHit}(\textit{self}, \textit{beam})

    \vspace{-1.5ex}

    \rule{\textwidth}{0.5\fboxrule}
\setlength{\parskip}{2ex}
\begin{alltt}
Determine if a beam hits the BeamDump.

This uses the line***Inter functions from the geometry module to find
characteristics of impact of beams on beamdumps.

beam: incoming beam. [GaussianBeam]

Returns a dictionary with keys:
    'isHit': whether the beam hits the dump. [boolean]
    'intersection point': point in space where it is first hit.
        [3D vector]
    'face': to indicate which face is first hit, can be 'HR', 'AR' or
        'side'. [string]
    'distance': geometrical distance from beam origin to impact. [float]
\end{alltt}

\setlength{\parskip}{1ex}
      Overrides: theia.optics.component.SetupComponent.isHit

    \end{boxedminipage}

    \vspace{0.5ex}

\hspace{.8\funcindent}\begin{boxedminipage}{\funcwidth}

    \raggedright \textbf{hit}(\textit{self}, \textit{beam}, \textit{order}, \textit{threshold})

    \vspace{-1.5ex}

    \rule{\textwidth}{0.5\fboxrule}
\setlength{\parskip}{2ex}
\begin{alltt}
Compute the refracted and reflected beams after interaction.

BeamDumps always stop beams.

beam: incident beam. [GaussianBeam]
order: maximum strayness of daughter beams, which are not returned if
    their strayness is over this order. [integer]
threshold: idem for the power of the daughter beams. [float]

Returns a dictionary of beams with keys:
    't': None
    'r': None
\end{alltt}

\setlength{\parskip}{1ex}
      Overrides: theia.optics.component.SetupComponent.hit

    \end{boxedminipage}


\large{\textbf{\textit{Inherited from theia.optics.component.SetupComponent\textit{(Section \ref{theia:optics:component:SetupComponent})}}}}

\begin{quote}
\_\_str\_\_(), translate()
\end{quote}

\large{\textbf{\textit{Inherited from object}}}

\begin{quote}
\_\_delattr\_\_(), \_\_format\_\_(), \_\_getattribute\_\_(), \_\_hash\_\_(), \_\_new\_\_(), \_\_reduce\_\_(), \_\_reduce\_ex\_\_(), \_\_repr\_\_(), \_\_setattr\_\_(), \_\_sizeof\_\_(), \_\_subclasshook\_\_()
\end{quote}

%%%%%%%%%%%%%%%%%%%%%%%%%%%%%%%%%%%%%%%%%%%%%%%%%%%%%%%%%%%%%%%%%%%%%%%%%%%
%%                              Properties                               %%
%%%%%%%%%%%%%%%%%%%%%%%%%%%%%%%%%%%%%%%%%%%%%%%%%%%%%%%%%%%%%%%%%%%%%%%%%%%

  \subsubsection{Properties}

    \vspace{-1cm}
\hspace{\varindent}\begin{longtable}{|p{\varnamewidth}|p{\vardescrwidth}|l}
\cline{1-2}
\cline{1-2} \centering \textbf{Name} & \centering \textbf{Description}& \\
\cline{1-2}
\endhead\cline{1-2}\multicolumn{3}{r}{\small\textit{continued on next page}}\\\endfoot\cline{1-2}
\endlastfoot\multicolumn{2}{|l|}{\textit{Inherited from object}}\\
\multicolumn{2}{|p{\varwidth}|}{\raggedright \_\_class\_\_}\\
\cline{1-2}
\end{longtable}


%%%%%%%%%%%%%%%%%%%%%%%%%%%%%%%%%%%%%%%%%%%%%%%%%%%%%%%%%%%%%%%%%%%%%%%%%%%
%%                            Class Variables                            %%
%%%%%%%%%%%%%%%%%%%%%%%%%%%%%%%%%%%%%%%%%%%%%%%%%%%%%%%%%%%%%%%%%%%%%%%%%%%

  \subsubsection{Class Variables}

    \vspace{-1cm}
\hspace{\varindent}\begin{longtable}{|p{\varnamewidth}|p{\vardescrwidth}|l}
\cline{1-2}
\cline{1-2} \centering \textbf{Name} & \centering \textbf{Description}& \\
\cline{1-2}
\endhead\cline{1-2}\multicolumn{3}{r}{\small\textit{continued on next page}}\\\endfoot\cline{1-2}
\endlastfoot\raggedright N\-a\-m\-e\- & \raggedright \textbf{Value:} 
{\tt "BeamDump"}&\\
\cline{1-2}
\multicolumn{2}{|l|}{\textit{Inherited from theia.optics.component.SetupComponent \textit{(Section \ref{theia:optics:component:SetupComponent})}}}\\
\multicolumn{2}{|p{\varwidth}|}{\raggedright SetupCount, \_\_abstractmethods\_\_}\\
\cline{1-2}
\end{longtable}

    \index{theia \textit{(package)}!theia.optics \textit{(package)}!theia.optics.beamdump \textit{(module)}!theia.optics.beamdump.BeamDump \textit{(class)}|)}
    \index{theia \textit{(package)}!theia.optics \textit{(package)}!theia.optics.beamdump \textit{(module)}|)}
