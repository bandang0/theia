%
% API Documentation for theia
% Module theia.optics.component
%
% Generated by epydoc 3.0.1
% [Mon Jul 24 12:23:27 2017]
%

%%%%%%%%%%%%%%%%%%%%%%%%%%%%%%%%%%%%%%%%%%%%%%%%%%%%%%%%%%%%%%%%%%%%%%%%%%%
%%                          Module Description                           %%
%%%%%%%%%%%%%%%%%%%%%%%%%%%%%%%%%%%%%%%%%%%%%%%%%%%%%%%%%%%%%%%%%%%%%%%%%%%

    \index{theia \textit{(package)}!theia.optics \textit{(package)}!theia.optics.component \textit{(module)}|(}
\section{Module theia.optics.component}

    \label{theia:optics:component}
Defines the SetupComponent class for theia.


%%%%%%%%%%%%%%%%%%%%%%%%%%%%%%%%%%%%%%%%%%%%%%%%%%%%%%%%%%%%%%%%%%%%%%%%%%%
%%                               Variables                               %%
%%%%%%%%%%%%%%%%%%%%%%%%%%%%%%%%%%%%%%%%%%%%%%%%%%%%%%%%%%%%%%%%%%%%%%%%%%%

  \subsection{Variables}

    \vspace{-1cm}
\hspace{\varindent}\begin{longtable}{|p{\varnamewidth}|p{\vardescrwidth}|l}
\cline{1-2}
\cline{1-2} \centering \textbf{Name} & \centering \textbf{Description}& \\
\cline{1-2}
\endhead\cline{1-2}\multicolumn{3}{r}{\small\textit{continued on next page}}\\\endfoot\cline{1-2}
\endlastfoot\raggedright \_\-\_\-p\-a\-c\-k\-a\-g\-e\-\_\-\_\- & \raggedright \textbf{Value:} 
{\tt \texttt{'}\texttt{theia.optics}\texttt{'}}&\\
\cline{1-2}
\end{longtable}


%%%%%%%%%%%%%%%%%%%%%%%%%%%%%%%%%%%%%%%%%%%%%%%%%%%%%%%%%%%%%%%%%%%%%%%%%%%
%%                           Class Description                           %%
%%%%%%%%%%%%%%%%%%%%%%%%%%%%%%%%%%%%%%%%%%%%%%%%%%%%%%%%%%%%%%%%%%%%%%%%%%%

    \index{theia \textit{(package)}!theia.optics \textit{(package)}!theia.optics.component \textit{(module)}!theia.optics.component.SetupComponent \textit{(class)}|(}
\subsection{Class SetupComponent}

    \label{theia:optics:component:SetupComponent}
\begin{tabular}{cccccc}
% Line for object, linespec=[False]
\multicolumn{2}{r}{\settowidth{\BCL}{object}\multirow{2}{\BCL}{object}}
&&
  \\\cline{3-3}
  &&\multicolumn{1}{c|}{}
&&
  \\
&&\multicolumn{2}{l}{\textbf{theia.optics.component.SetupComponent}}
\end{tabular}

\textbf{Known Subclasses:}
theia.optics.beamdump.BeamDump,
    theia.optics.optic.Optic,
    theia.optics.ghost.Ghost

\begin{alltt}


SetupComponent class.

This is an Abstract Base Class for all the components (optical or not) of
the setup. Its methods may be implemented in daughter classes.

*=== Attributes ===*
SetupCount: class attribute, counts setup components. [integer]
HRCenter: center of the principal face of the component in space.
    [3D vector]
HRNorm: normal unitary vector the this principal face, supposed to point
    outside the media. [3D vector]
Thick: thickness of the component, counted in opposite direction to
    HRNorm. [float]
Dia: diameter of the component. [float]
Name: name of the component. [string]
Ref: reference string (for keeping track with the lab). [string]
\end{alltt}


%%%%%%%%%%%%%%%%%%%%%%%%%%%%%%%%%%%%%%%%%%%%%%%%%%%%%%%%%%%%%%%%%%%%%%%%%%%
%%                                Methods                                %%
%%%%%%%%%%%%%%%%%%%%%%%%%%%%%%%%%%%%%%%%%%%%%%%%%%%%%%%%%%%%%%%%%%%%%%%%%%%

  \subsubsection{Methods}

    \vspace{0.5ex}

\hspace{.8\funcindent}\begin{boxedminipage}{\funcwidth}

    \raggedright \textbf{\_\_init\_\_}(\textit{self}, \textit{HRCenter}, \textit{HRNorm}, \textit{Ref}, \textit{Thickness}, \textit{Diameter})

    \vspace{-1.5ex}

    \rule{\textwidth}{0.5\fboxrule}
\setlength{\parskip}{2ex}
    SetupComponent initializer.

    Parameters are the attributes of the object to construct.

    Returns a setupComponent.

\setlength{\parskip}{1ex}
      Overrides: object.\_\_init\_\_

    \end{boxedminipage}

    \vspace{0.5ex}

\hspace{.8\funcindent}\begin{boxedminipage}{\funcwidth}

    \raggedright \textbf{\_\_str\_\_}(\textit{self})

    \vspace{-1.5ex}

    \rule{\textwidth}{0.5\fboxrule}
\setlength{\parskip}{2ex}
    String representation of the component, when calling print(object).

\setlength{\parskip}{1ex}
      Overrides: object.\_\_str\_\_

    \end{boxedminipage}

    \label{theia:optics:component:SetupComponent:hit}
    \index{theia \textit{(package)}!theia.optics \textit{(package)}!theia.optics.component \textit{(module)}!theia.optics.component.SetupComponent \textit{(class)}!theia.optics.component.SetupComponent.hit \textit{(method)}}

    \vspace{0.5ex}

\hspace{.8\funcindent}\begin{boxedminipage}{\funcwidth}

    \raggedright \textbf{hit}(\textit{self}, \textit{beam}, \textit{order}, \textit{threshold})

    \vspace{-1.5ex}

    \rule{\textwidth}{0.5\fboxrule}
\setlength{\parskip}{2ex}
    Compute the refracted and reflected beams after interaction.

    Abstract (pure virtual) method.

\setlength{\parskip}{1ex}
    \end{boxedminipage}

    \label{theia:optics:component:SetupComponent:isHit}
    \index{theia \textit{(package)}!theia.optics \textit{(package)}!theia.optics.component \textit{(module)}!theia.optics.component.SetupComponent \textit{(class)}!theia.optics.component.SetupComponent.isHit \textit{(method)}}

    \vspace{0.5ex}

\hspace{.8\funcindent}\begin{boxedminipage}{\funcwidth}

    \raggedright \textbf{isHit}(\textit{self}, \textit{beam})

    \vspace{-1.5ex}

    \rule{\textwidth}{0.5\fboxrule}
\setlength{\parskip}{2ex}
    Method to determine if component is hit by a beam.

    Abstract (pure virtual) method.

\setlength{\parskip}{1ex}
    \end{boxedminipage}

    \label{theia:optics:component:SetupComponent:lines}
    \index{theia \textit{(package)}!theia.optics \textit{(package)}!theia.optics.component \textit{(module)}!theia.optics.component.SetupComponent \textit{(class)}!theia.optics.component.SetupComponent.lines \textit{(method)}}

    \vspace{0.5ex}

\hspace{.8\funcindent}\begin{boxedminipage}{\funcwidth}

    \raggedright \textbf{lines}(\textit{self})

    \vspace{-1.5ex}

    \rule{\textwidth}{0.5\fboxrule}
\setlength{\parskip}{2ex}
    Method to return the list of strings to \_\_str\_\_.

    Abstract (pure virtual) method.

\setlength{\parskip}{1ex}
    \end{boxedminipage}

    \label{theia:optics:component:SetupComponent:translate}
    \index{theia \textit{(package)}!theia.optics \textit{(package)}!theia.optics.component \textit{(module)}!theia.optics.component.SetupComponent \textit{(class)}!theia.optics.component.SetupComponent.translate \textit{(method)}}

    \vspace{0.5ex}

\hspace{.8\funcindent}\begin{boxedminipage}{\funcwidth}

    \raggedright \textbf{translate}(\textit{self}, \textit{X}={\tt 0.0}, \textit{Y}={\tt 0.0}, \textit{Z}={\tt 0.0})

    \vspace{-1.5ex}

    \rule{\textwidth}{0.5\fboxrule}
\setlength{\parskip}{2ex}
    Move the component to (current position + (X, Y, Z)).

    This version only takes care of the HRCenter, version of sub classes 
    take care of ARCenter if relevant.

    X, Y, Z: components of the translation vector.

    No return value.

\setlength{\parskip}{1ex}
    \end{boxedminipage}


\large{\textbf{\textit{Inherited from object}}}

\begin{quote}
\_\_delattr\_\_(), \_\_format\_\_(), \_\_getattribute\_\_(), \_\_hash\_\_(), \_\_new\_\_(), \_\_reduce\_\_(), \_\_reduce\_ex\_\_(), \_\_repr\_\_(), \_\_setattr\_\_(), \_\_sizeof\_\_(), \_\_subclasshook\_\_()
\end{quote}

%%%%%%%%%%%%%%%%%%%%%%%%%%%%%%%%%%%%%%%%%%%%%%%%%%%%%%%%%%%%%%%%%%%%%%%%%%%
%%                              Properties                               %%
%%%%%%%%%%%%%%%%%%%%%%%%%%%%%%%%%%%%%%%%%%%%%%%%%%%%%%%%%%%%%%%%%%%%%%%%%%%

  \subsubsection{Properties}

    \vspace{-1cm}
\hspace{\varindent}\begin{longtable}{|p{\varnamewidth}|p{\vardescrwidth}|l}
\cline{1-2}
\cline{1-2} \centering \textbf{Name} & \centering \textbf{Description}& \\
\cline{1-2}
\endhead\cline{1-2}\multicolumn{3}{r}{\small\textit{continued on next page}}\\\endfoot\cline{1-2}
\endlastfoot\multicolumn{2}{|l|}{\textit{Inherited from object}}\\
\multicolumn{2}{|p{\varwidth}|}{\raggedright \_\_class\_\_}\\
\cline{1-2}
\end{longtable}


%%%%%%%%%%%%%%%%%%%%%%%%%%%%%%%%%%%%%%%%%%%%%%%%%%%%%%%%%%%%%%%%%%%%%%%%%%%
%%                            Class Variables                            %%
%%%%%%%%%%%%%%%%%%%%%%%%%%%%%%%%%%%%%%%%%%%%%%%%%%%%%%%%%%%%%%%%%%%%%%%%%%%

  \subsubsection{Class Variables}

    \vspace{-1cm}
\hspace{\varindent}\begin{longtable}{|p{\varnamewidth}|p{\vardescrwidth}|l}
\cline{1-2}
\cline{1-2} \centering \textbf{Name} & \centering \textbf{Description}& \\
\cline{1-2}
\endhead\cline{1-2}\multicolumn{3}{r}{\small\textit{continued on next page}}\\\endfoot\cline{1-2}
\endlastfoot\raggedright N\-a\-m\-e\- & \raggedright \textbf{Value:} 
{\tt \texttt{'}\texttt{SetupComponent}\texttt{'}}&\\
\cline{1-2}
\raggedright S\-e\-t\-u\-p\-C\-o\-u\-n\-t\- & \raggedright \textbf{Value:} 
{\tt 0}&\\
\cline{1-2}
\raggedright \_\-\_\-a\-b\-s\-t\-r\-a\-c\-t\-m\-e\-t\-h\-o\-d\-s\-\_\-\_\- & \raggedright \textbf{Value:} 
{\tt \texttt{frozenset([}\texttt{'}\texttt{hit}\texttt{'}\texttt{, }\texttt{'}\texttt{isHit}\texttt{'}\texttt{, }\texttt{'}\texttt{lines}\texttt{'}\texttt{])}}&\\
\cline{1-2}
\end{longtable}

    \index{theia \textit{(package)}!theia.optics \textit{(package)}!theia.optics.component \textit{(module)}!theia.optics.component.SetupComponent \textit{(class)}|)}
    \index{theia \textit{(package)}!theia.optics \textit{(package)}!theia.optics.component \textit{(module)}|)}
