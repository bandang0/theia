%
% API Documentation for theia
% Package theia
%
% Generated by epydoc 3.0.1
% [Thu Aug 24 10:17:28 2017]
%

%%%%%%%%%%%%%%%%%%%%%%%%%%%%%%%%%%%%%%%%%%%%%%%%%%%%%%%%%%%%%%%%%%%%%%%%%%%
%%                          Module Description                           %%
%%%%%%%%%%%%%%%%%%%%%%%%%%%%%%%%%%%%%%%%%%%%%%%%%%%%%%%%%%%%%%%%%%%%%%%%%%%

    \index{theia \textit{(package)}|(}
\section{Package theia}

    \label{theia}
This is theia, a Python package for Gaussian ray tracing in 3D optical 
setups.

\textbf{Version:} 0.1.3



\textbf{Author:} Raphaël Duque



\textbf{Copyright:} Copyright 2017, Raphaël Duque



\textbf{License:} GNU GPLv3+




%%%%%%%%%%%%%%%%%%%%%%%%%%%%%%%%%%%%%%%%%%%%%%%%%%%%%%%%%%%%%%%%%%%%%%%%%%%
%%                                Modules                                %%
%%%%%%%%%%%%%%%%%%%%%%%%%%%%%%%%%%%%%%%%%%%%%%%%%%%%%%%%%%%%%%%%%%%%%%%%%%%

\subsection{Modules}

\begin{itemize}
\setlength{\parskip}{0ex}
\item \textbf{helpers}: This is the helpers sub-package of theia.



  \textit{(Section \ref{theia:helpers}, p.~\pageref{theia:helpers})}

  \begin{itemize}
\setlength{\parskip}{0ex}
    \item \textbf{core}: Defines some additional spice for theia.



  \textit{(Section \ref{theia:helpers:core}, p.~\pageref{theia:helpers:core})}

    \item \textbf{geometry}: Geometry module for theia.



  \textit{(Section \ref{theia:helpers:geometry}, p.~\pageref{theia:helpers:geometry})}

    \item \textbf{interaction}: Module to define interaction functions for theia.



  \textit{(Section \ref{theia:helpers:interaction}, p.~\pageref{theia:helpers:interaction})}

    \item \textbf{settings}: Module to initiate all global variables for theia.



  \textit{(Section \ref{theia:helpers:settings}, p.~\pageref{theia:helpers:settings})}

    \item \textbf{tools}: Defines some generic functions for theia.



  \textit{(Section \ref{theia:helpers:tools}, p.~\pageref{theia:helpers:tools})}

    \item \textbf{units}: Various units for theia.



  \textit{(Section \ref{theia:helpers:units}, p.~\pageref{theia:helpers:units})}

  \end{itemize}
\item \textbf{main}: Main module of theia, defines the main function.



  \textit{(Section \ref{theia:main}, p.~\pageref{theia:main})}

\item \textbf{optics}: This is the optics sub-package of theia.



  \textit{(Section \ref{theia:optics}, p.~\pageref{theia:optics})}

  \begin{itemize}
\setlength{\parskip}{0ex}
    \item \textbf{beam}: Defines the GaussianBeam class for theia.



  \textit{(Section \ref{theia:optics:beam}, p.~\pageref{theia:optics:beam})}

    \item \textbf{beamdump}: Defines the BeamDump class for theia.



  \textit{(Section \ref{theia:optics:beamdump}, p.~\pageref{theia:optics:beamdump})}

    \item \textbf{beamsplitter}: Defines the BeamSplitter class for theia.



  \textit{(Section \ref{theia:optics:beamsplitter}, p.~\pageref{theia:optics:beamsplitter})}

    \item \textbf{component}: Defines the SetupComponent class for theia.



  \textit{(Section \ref{theia:optics:component}, p.~\pageref{theia:optics:component})}

    \item \textbf{ghost}: Defines the Ghost class for theia.



  \textit{(Section \ref{theia:optics:ghost}, p.~\pageref{theia:optics:ghost})}

    \item \textbf{mirror}: Defines the Mirror class for theia.



  \textit{(Section \ref{theia:optics:mirror}, p.~\pageref{theia:optics:mirror})}

    \item \textbf{optic}: Defines the Optic class for theia.



  \textit{(Section \ref{theia:optics:optic}, p.~\pageref{theia:optics:optic})}

    \item \textbf{special}: Defines the Special class for theia.



  \textit{(Section \ref{theia:optics:special}, p.~\pageref{theia:optics:special})}

    \item \textbf{thicklens}: Defines the ThickLens class for theia.



  \textit{(Section \ref{theia:optics:thicklens}, p.~\pageref{theia:optics:thicklens})}

    \item \textbf{thinlens}: Defines the ThinLens class for theia.



  \textit{(Section \ref{theia:optics:thinlens}, p.~\pageref{theia:optics:thinlens})}

  \end{itemize}
\item \textbf{rendering}: This is the rendering sub-package of theia.



  \textit{(Section \ref{theia:rendering}, p.~\pageref{theia:rendering})}

  \begin{itemize}
\setlength{\parskip}{0ex}
    \item \textbf{features}: Features module or theia, to represent objects as FreeCAD Python features.



  \textit{(Section \ref{theia:rendering:features}, p.~\pageref{theia:rendering:features})}

    \item \textbf{shapes}: Shapes module for theia, provides shape-calculating for 3D rendering.



  \textit{(Section \ref{theia:rendering:shapes}, p.~\pageref{theia:rendering:shapes})}

    \item \textbf{writer}: Writer module for theia, to write CAD content to files.



  \textit{(Section \ref{theia:rendering:writer}, p.~\pageref{theia:rendering:writer})}

  \end{itemize}
\item \textbf{running}: This is the running sub-package of theia.



  \textit{(Section \ref{theia:running}, p.~\pageref{theia:running})}

  \begin{itemize}
\setlength{\parskip}{0ex}
    \item \textbf{parser}: Module for the parsing on input data from .tia file.



  \textit{(Section \ref{theia:running:parser}, p.~\pageref{theia:running:parser})}

    \item \textbf{simulation}: Defines the Simulation class for theia.



  \textit{(Section \ref{theia:running:simulation}, p.~\pageref{theia:running:simulation})}

  \end{itemize}
\item \textbf{tree}: This is the tree sub-package of theia.



  \textit{(Section \ref{theia:tree}, p.~\pageref{theia:tree})}

  \begin{itemize}
\setlength{\parskip}{0ex}
    \item \textbf{beamtree}: Defines the BeamTree class for theia.



  \textit{(Section \ref{theia:tree:beamtree}, p.~\pageref{theia:tree:beamtree})}

  \end{itemize}
\end{itemize}

    \index{theia \textit{(package)}|)}
