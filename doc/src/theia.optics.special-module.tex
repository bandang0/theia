%
% API Documentation for theia
% Module theia.optics.special
%
% Generated by epydoc 3.0.1
% [Fri Aug 18 08:32:31 2017]
%

%%%%%%%%%%%%%%%%%%%%%%%%%%%%%%%%%%%%%%%%%%%%%%%%%%%%%%%%%%%%%%%%%%%%%%%%%%%
%%                          Module Description                           %%
%%%%%%%%%%%%%%%%%%%%%%%%%%%%%%%%%%%%%%%%%%%%%%%%%%%%%%%%%%%%%%%%%%%%%%%%%%%

    \index{theia \textit{(package)}!theia.optics \textit{(package)}!theia.optics.special \textit{(module)}|(}
\section{Module theia.optics.special}

    \label{theia:optics:special}
Defines the Special class for theia.


%%%%%%%%%%%%%%%%%%%%%%%%%%%%%%%%%%%%%%%%%%%%%%%%%%%%%%%%%%%%%%%%%%%%%%%%%%%
%%                           Class Description                           %%
%%%%%%%%%%%%%%%%%%%%%%%%%%%%%%%%%%%%%%%%%%%%%%%%%%%%%%%%%%%%%%%%%%%%%%%%%%%

    \index{theia \textit{(package)}!theia.optics \textit{(package)}!theia.optics.special \textit{(module)}!theia.optics.special.Special \textit{(class)}|(}
\subsection{Class Special}

    \label{theia:optics:special:Special}
\begin{tabular}{cccccccccc}
% Line for object, linespec=[False, False, False]
\multicolumn{2}{r}{\settowidth{\BCL}{object}\multirow{2}{\BCL}{object}}
&&
&&
&&
  \\\cline{3-3}
  &&\multicolumn{1}{c|}{}
&&
&&
&&
  \\
% Line for theia.optics.component.SetupComponent, linespec=[False, False]
\multicolumn{4}{r}{\settowidth{\BCL}{theia.optics.component.SetupComponent}\multirow{2}{\BCL}{theia.optics.component.SetupComponent}}
&&
&&
  \\\cline{5-5}
  &&&&\multicolumn{1}{c|}{}
&&
&&
  \\
% Line for theia.optics.optic.Optic, linespec=[False]
\multicolumn{6}{r}{\settowidth{\BCL}{theia.optics.optic.Optic}\multirow{2}{\BCL}{theia.optics.optic.Optic}}
&&
  \\\cline{7-7}
  &&&&&&\multicolumn{1}{c|}{}
&&
  \\
&&&&&&\multicolumn{2}{l}{\textbf{theia.optics.special.Special}}
\end{tabular}

\begin{alltt}


Special class.

This class represents general optics, as their actions on R and T are left
to the user to input. They are useful for special optics which are neither
reflective nor transmissive.

Actions:
    * T on HR: user input
    * R on HR: user input
    * T on AR: user input
    * R on AR: user input

**Note**: by default the actions of these objects are those of
beamsplitters (0, 0, 0, 0)

*=== Additional attributes with respect to the Optic class ===*

None

*=== Name ===*

Special

**Note**: the curvature of any surface is positive for a concave surface
(coating inside the sphere).
Thus kurv*HRNorm/{\textbar}kurv{\textbar} always points to the center
of the sphere of the surface, as is the convention for the lineSurfInter of
geometry module. Same for AR.

*******     HRK {\textgreater} 0 and ARK {\textgreater} 0     *******           HRK {\textgreater} 0 and ARK {\textless} 0
 *****                               ********         and {\textbar}ARK{\textbar} {\textgreater} {\textbar}HRK{\textbar}
 H***A                               H*********A
 *****                               ********
*******                             *******
\end{alltt}


%%%%%%%%%%%%%%%%%%%%%%%%%%%%%%%%%%%%%%%%%%%%%%%%%%%%%%%%%%%%%%%%%%%%%%%%%%%
%%                                Methods                                %%
%%%%%%%%%%%%%%%%%%%%%%%%%%%%%%%%%%%%%%%%%%%%%%%%%%%%%%%%%%%%%%%%%%%%%%%%%%%

  \subsubsection{Methods}

    \vspace{0.5ex}

\hspace{.8\funcindent}\begin{boxedminipage}{\funcwidth}

    \raggedright \textbf{\_\_init\_\_}(\textit{self}, \textit{Wedge}={\tt 0.}, \textit{Alpha}={\tt 0.}, \textit{X}={\tt 0.}, \textit{Y}={\tt 0.}, \textit{Z}={\tt 0.}, \textit{Theta}={\tt pi/2.}, \textit{Phi}={\tt 0.}, \textit{Diameter}={\tt 10.e-2}, \textit{HRr}={\tt .99}, \textit{HRt}={\tt .01}, \textit{ARr}={\tt .1}, \textit{ARt}={\tt .9}, \textit{HRK}={\tt 0.01}, \textit{ARK}={\tt 0}, \textit{Thickness}={\tt 2.e-2}, \textit{N}={\tt 1.4585}, \textit{KeepI}={\tt False}, \textit{RonHR}={\tt 0}, \textit{TonHR}={\tt 0}, \textit{RonAR}={\tt 0}, \textit{TonAR}={\tt 0}, \textit{Ref}={\tt None})

    \vspace{-1.5ex}

    \rule{\textwidth}{0.5\fboxrule}
\setlength{\parskip}{2ex}
    Special optic initializer.

    Parameters are the attributes.

    Returns a special optic.

\setlength{\parskip}{1ex}
      Overrides: object.\_\_init\_\_

    \end{boxedminipage}

    \vspace{0.5ex}

\hspace{.8\funcindent}\begin{boxedminipage}{\funcwidth}

    \raggedright \textbf{lines}(\textit{self})

    \vspace{-1.5ex}

    \rule{\textwidth}{0.5\fboxrule}
\setlength{\parskip}{2ex}
    Returns the list of lines necessary to print the object.

\setlength{\parskip}{1ex}
      Overrides: theia.optics.component.SetupComponent.lines

    \end{boxedminipage}


\large{\textbf{\textit{Inherited from theia.optics.optic.Optic\textit{(Section \ref{theia:optics:optic:Optic})}}}}

\begin{quote}
apexes(), collision(), geoCheck(), hit(), hitAR(), hitHR(), hitSide(), isHit(), isHitDics(), translate()
\end{quote}

\large{\textbf{\textit{Inherited from theia.optics.component.SetupComponent\textit{(Section \ref{theia:optics:component:SetupComponent})}}}}

\begin{quote}
\_\_str\_\_()
\end{quote}

\large{\textbf{\textit{Inherited from object}}}

\begin{quote}
\_\_delattr\_\_(), \_\_format\_\_(), \_\_getattribute\_\_(), \_\_hash\_\_(), \_\_new\_\_(), \_\_reduce\_\_(), \_\_reduce\_ex\_\_(), \_\_repr\_\_(), \_\_setattr\_\_(), \_\_sizeof\_\_(), \_\_subclasshook\_\_()
\end{quote}

%%%%%%%%%%%%%%%%%%%%%%%%%%%%%%%%%%%%%%%%%%%%%%%%%%%%%%%%%%%%%%%%%%%%%%%%%%%
%%                              Properties                               %%
%%%%%%%%%%%%%%%%%%%%%%%%%%%%%%%%%%%%%%%%%%%%%%%%%%%%%%%%%%%%%%%%%%%%%%%%%%%

  \subsubsection{Properties}

    \vspace{-1cm}
\hspace{\varindent}\begin{longtable}{|p{\varnamewidth}|p{\vardescrwidth}|l}
\cline{1-2}
\cline{1-2} \centering \textbf{Name} & \centering \textbf{Description}& \\
\cline{1-2}
\endhead\cline{1-2}\multicolumn{3}{r}{\small\textit{continued on next page}}\\\endfoot\cline{1-2}
\endlastfoot\multicolumn{2}{|l|}{\textit{Inherited from object}}\\
\multicolumn{2}{|p{\varwidth}|}{\raggedright \_\_class\_\_}\\
\cline{1-2}
\end{longtable}


%%%%%%%%%%%%%%%%%%%%%%%%%%%%%%%%%%%%%%%%%%%%%%%%%%%%%%%%%%%%%%%%%%%%%%%%%%%
%%                            Class Variables                            %%
%%%%%%%%%%%%%%%%%%%%%%%%%%%%%%%%%%%%%%%%%%%%%%%%%%%%%%%%%%%%%%%%%%%%%%%%%%%

  \subsubsection{Class Variables}

    \vspace{-1cm}
\hspace{\varindent}\begin{longtable}{|p{\varnamewidth}|p{\vardescrwidth}|l}
\cline{1-2}
\cline{1-2} \centering \textbf{Name} & \centering \textbf{Description}& \\
\cline{1-2}
\endhead\cline{1-2}\multicolumn{3}{r}{\small\textit{continued on next page}}\\\endfoot\cline{1-2}
\endlastfoot\raggedright N\-a\-m\-e\- & \raggedright \textbf{Value:} 
{\tt "Special"}&\\
\cline{1-2}
\multicolumn{2}{|l|}{\textit{Inherited from theia.optics.optic.Optic \textit{(Section \ref{theia:optics:optic:Optic})}}}\\
\multicolumn{2}{|p{\varwidth}|}{\raggedright OptCount}\\
\cline{1-2}
\multicolumn{2}{|l|}{\textit{Inherited from theia.optics.component.SetupComponent \textit{(Section \ref{theia:optics:component:SetupComponent})}}}\\
\multicolumn{2}{|p{\varwidth}|}{\raggedright SetupCount, \_\_abstractmethods\_\_}\\
\cline{1-2}
\end{longtable}

    \index{theia \textit{(package)}!theia.optics \textit{(package)}!theia.optics.special \textit{(module)}!theia.optics.special.Special \textit{(class)}|)}
    \index{theia \textit{(package)}!theia.optics \textit{(package)}!theia.optics.special \textit{(module)}|)}
