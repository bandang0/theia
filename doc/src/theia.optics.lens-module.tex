%
% API Documentation for theia
% Module theia.optics.lens
%
% Generated by epydoc 3.0.1
% [Tue Jul 11 10:03:01 2017]
%

%%%%%%%%%%%%%%%%%%%%%%%%%%%%%%%%%%%%%%%%%%%%%%%%%%%%%%%%%%%%%%%%%%%%%%%%%%%
%%                          Module Description                           %%
%%%%%%%%%%%%%%%%%%%%%%%%%%%%%%%%%%%%%%%%%%%%%%%%%%%%%%%%%%%%%%%%%%%%%%%%%%%

    \index{theia \textit{(package)}!theia.optics \textit{(package)}!theia.optics.lens \textit{(module)}|(}
\section{Module theia.optics.lens}

    \label{theia:optics:lens}
Defines the Lens class for theia.


%%%%%%%%%%%%%%%%%%%%%%%%%%%%%%%%%%%%%%%%%%%%%%%%%%%%%%%%%%%%%%%%%%%%%%%%%%%
%%                               Variables                               %%
%%%%%%%%%%%%%%%%%%%%%%%%%%%%%%%%%%%%%%%%%%%%%%%%%%%%%%%%%%%%%%%%%%%%%%%%%%%

  \subsection{Variables}

    \vspace{-1cm}
\hspace{\varindent}\begin{longtable}{|p{\varnamewidth}|p{\vardescrwidth}|l}
\cline{1-2}
\cline{1-2} \centering \textbf{Name} & \centering \textbf{Description}& \\
\cline{1-2}
\endhead\cline{1-2}\multicolumn{3}{r}{\small\textit{continued on next page}}\\\endfoot\cline{1-2}
\endlastfoot\raggedright \_\-\_\-p\-a\-c\-k\-a\-g\-e\-\_\-\_\- & \raggedright \textbf{Value:} 
{\tt \texttt{'}\texttt{theia.optics}\texttt{'}}&\\
\cline{1-2}
\end{longtable}


%%%%%%%%%%%%%%%%%%%%%%%%%%%%%%%%%%%%%%%%%%%%%%%%%%%%%%%%%%%%%%%%%%%%%%%%%%%
%%                           Class Description                           %%
%%%%%%%%%%%%%%%%%%%%%%%%%%%%%%%%%%%%%%%%%%%%%%%%%%%%%%%%%%%%%%%%%%%%%%%%%%%

    \index{theia \textit{(package)}!theia.optics \textit{(package)}!theia.optics.lens \textit{(module)}!theia.optics.lens.Lens \textit{(class)}|(}
\subsection{Class Lens}

    \label{theia:optics:lens:Lens}
\begin{tabular}{cccccccccc}
% Line for object, linespec=[False, False, False]
\multicolumn{2}{r}{\settowidth{\BCL}{object}\multirow{2}{\BCL}{object}}
&&
&&
&&
  \\\cline{3-3}
  &&\multicolumn{1}{c|}{}
&&
&&
&&
  \\
% Line for theia.optics.component.SetupComponent, linespec=[False, False]
\multicolumn{4}{r}{\settowidth{\BCL}{theia.optics.component.SetupComponent}\multirow{2}{\BCL}{theia.optics.component.SetupComponent}}
&&
&&
  \\\cline{5-5}
  &&&&\multicolumn{1}{c|}{}
&&
&&
  \\
% Line for theia.optics.optic.Optic, linespec=[False]
\multicolumn{6}{r}{\settowidth{\BCL}{theia.optics.optic.Optic}\multirow{2}{\BCL}{theia.optics.optic.Optic}}
&&
  \\\cline{7-7}
  &&&&&&\multicolumn{1}{c|}{}
&&
  \\
&&&&&&\multicolumn{2}{l}{\textbf{theia.optics.lens.Lens}}
\end{tabular}

\textbf{Known Subclasses:}
theia.optics.thicklens.ThickLens,
    theia.optics.thinlens.ThinLens

\begin{alltt}


Lens class.

This class is a base class for lenses. It implements the hit and hitActive
methods for all lenses.

*=== Attributes ===*
SetupCount (inherited): class attribute, counts all setup components.
    [integer]
OptCount (inherited): class attribute, counts optical components. [integer]
HRCenter (inherited): center of the 'chord' of the HR surface. [3D vector]
HRNorm (inherited): unitary normal to the 'chord' of the HR (always pointing
    towards the outside of the component). [3D vector]
Thick (inherited): thickness of the optic, counted in opposite direction to
    HRNorm. [float]
Dia (inherited): diameter of the component. [float]
Name (inherited): name of the component. [string]
Ref (inherited): reference string (for keeping track with the lab). [string]
ARCenter (inherited): center of the 'chord' of the AR surface. [3D vector]
ARNorm (inherited): unitary normal to the 'chord' of the AR (always pointing
    towards the outside of the component). [3D vector]
N (inherited): refraction index of the material. [float]
HRK, ARK (inherited): curvature of the HR, AR surfaces. [float]
HRr, HRt, ARr, ARt (inherited): power reflectance and transmission
    coefficients of the HR and AR surfaces. [float]
KeepI (inherited): whether of not to keep data of rays for interference
    calculations on the HR. [boolean]

**Note**: the curvature of any surface is positive for a concave surface
(coating inside the sphere).
Thus kurv*HRNorm/{\textbar}kurv{\textbar} always points to the center
of the sphere of the surface, as is the convention for the lineSurfInter of
geometry module. Same for AR.

*******     HRK {\textgreater} 0 and ARK {\textgreater} 0     *******           HRK {\textgreater} 0 and ARK {\textless} 0
 *****                               ********         and {\textbar}ARK{\textbar} {\textgreater} {\textbar}HRK{\textbar}
 H***A                               H*********A
 *****                               ********
*******                             *******
\end{alltt}


%%%%%%%%%%%%%%%%%%%%%%%%%%%%%%%%%%%%%%%%%%%%%%%%%%%%%%%%%%%%%%%%%%%%%%%%%%%
%%                                Methods                                %%
%%%%%%%%%%%%%%%%%%%%%%%%%%%%%%%%%%%%%%%%%%%%%%%%%%%%%%%%%%%%%%%%%%%%%%%%%%%

  \subsubsection{Methods}

    \vspace{0.5ex}

\hspace{.8\funcindent}\begin{boxedminipage}{\funcwidth}

    \raggedright \textbf{isHit}(\textit{self}, \textit{beam})

    \vspace{-1.5ex}

    \rule{\textwidth}{0.5\fboxrule}
\setlength{\parskip}{2ex}
\begin{alltt}
Determine if a beam hits the Lens.

This is a generic function for all lenses, using their geometrical
attributes. This uses the line***Inter functions from the geometry
module to find characteristics of impact of beams on lenses.

beam: incoming beam. [GaussianBeam]

Returns a dictionary with keys:
    'isHit': whether the beam hits the optic. [boolean]
    'intersection point': point in space where it is first hit.
            [3D vector]
    'face': to indicate which face is first hit, can be 'HR', 'AR' or
        'side'. [string]
    'distance': geometrical distance from beam origin to impact. [float]
\end{alltt}

\setlength{\parskip}{1ex}
      Overrides: theia.optics.component.SetupComponent.isHit

    \end{boxedminipage}

    \vspace{0.5ex}

\hspace{.8\funcindent}\begin{boxedminipage}{\funcwidth}

    \raggedright \textbf{hit}(\textit{self}, \textit{beam}, \textit{order}, \textit{threshold})

    \vspace{-1.5ex}

    \rule{\textwidth}{0.5\fboxrule}
\setlength{\parskip}{2ex}
\begin{alltt}
Compute the refracted and reflected beams after interaction.

This function is valid for all types of lenses.
The beams returned are those selected after the order and threshold
criterion.

beam: incident beam. [GaussianBeam]
order: maximum strayness of daughter beams, whixh are not returned if
    their strayness is over this order. [integer]
threshold: idem for the power of the daughter beams. [float]

Returns a dictionary of beams with keys:
    't': refracted beam. [GaussianBeam]
    'r': reflected beam. [GaussianBeam]
\end{alltt}

\setlength{\parskip}{1ex}
      Overrides: theia.optics.component.SetupComponent.hit

    \end{boxedminipage}

    \label{theia:optics:lens:Lens:hitActive}
    \index{theia \textit{(package)}!theia.optics \textit{(package)}!theia.optics.lens \textit{(module)}!theia.optics.lens.Lens \textit{(class)}!theia.optics.lens.Lens.hitActive \textit{(method)}}

    \vspace{0.5ex}

\hspace{.8\funcindent}\begin{boxedminipage}{\funcwidth}

    \raggedright \textbf{hitActive}(\textit{self}, \textit{beam}, \textit{point}, \textit{faceTag}, \textit{order}, \textit{threshold})

    \vspace{-1.5ex}

    \rule{\textwidth}{0.5\fboxrule}
\setlength{\parskip}{2ex}
\begin{alltt}
Compute the daughter beams after interaction on HR or AR at point.

AR andHr are the 'active' surfaces of the lens.
This function is valid for all types of lenses.

beam: incident beam. [GaussianBeam]
point: point in space of interaction. [3D vector]
faceTag: either 'AR' or 'HR' depending on the face. [string]
order: maximum strayness of daughter beams, whixh are not returned if
    their strayness is over this order. [integer]
threshold: idem for the power of the daughter beams. [float]

Returns a dictionary of beams with keys:
    't': refracted beam. [GaussianBeam]
    'r': reflected beam. [GaussianBeam]
\end{alltt}

\setlength{\parskip}{1ex}
    \end{boxedminipage}


\large{\textbf{\textit{Inherited from theia.optics.optic.Optic\textit{(Section \ref{theia:optics:optic:Optic})}}}}

\begin{quote}
\_\_init\_\_(), apexes(), collision(), geoCheck(), hitSide(), translate()
\end{quote}

\large{\textbf{\textit{Inherited from theia.optics.component.SetupComponent\textit{(Section \ref{theia:optics:component:SetupComponent})}}}}

\begin{quote}
\_\_str\_\_(), lines()
\end{quote}

\large{\textbf{\textit{Inherited from object}}}

\begin{quote}
\_\_delattr\_\_(), \_\_format\_\_(), \_\_getattribute\_\_(), \_\_hash\_\_(), \_\_new\_\_(), \_\_reduce\_\_(), \_\_reduce\_ex\_\_(), \_\_repr\_\_(), \_\_setattr\_\_(), \_\_sizeof\_\_(), \_\_subclasshook\_\_()
\end{quote}

%%%%%%%%%%%%%%%%%%%%%%%%%%%%%%%%%%%%%%%%%%%%%%%%%%%%%%%%%%%%%%%%%%%%%%%%%%%
%%                              Properties                               %%
%%%%%%%%%%%%%%%%%%%%%%%%%%%%%%%%%%%%%%%%%%%%%%%%%%%%%%%%%%%%%%%%%%%%%%%%%%%

  \subsubsection{Properties}

    \vspace{-1cm}
\hspace{\varindent}\begin{longtable}{|p{\varnamewidth}|p{\vardescrwidth}|l}
\cline{1-2}
\cline{1-2} \centering \textbf{Name} & \centering \textbf{Description}& \\
\cline{1-2}
\endhead\cline{1-2}\multicolumn{3}{r}{\small\textit{continued on next page}}\\\endfoot\cline{1-2}
\endlastfoot\multicolumn{2}{|l|}{\textit{Inherited from object}}\\
\multicolumn{2}{|p{\varwidth}|}{\raggedright \_\_class\_\_}\\
\cline{1-2}
\end{longtable}


%%%%%%%%%%%%%%%%%%%%%%%%%%%%%%%%%%%%%%%%%%%%%%%%%%%%%%%%%%%%%%%%%%%%%%%%%%%
%%                            Class Variables                            %%
%%%%%%%%%%%%%%%%%%%%%%%%%%%%%%%%%%%%%%%%%%%%%%%%%%%%%%%%%%%%%%%%%%%%%%%%%%%

  \subsubsection{Class Variables}

    \vspace{-1cm}
\hspace{\varindent}\begin{longtable}{|p{\varnamewidth}|p{\vardescrwidth}|l}
\cline{1-2}
\cline{1-2} \centering \textbf{Name} & \centering \textbf{Description}& \\
\cline{1-2}
\endhead\cline{1-2}\multicolumn{3}{r}{\small\textit{continued on next page}}\\\endfoot\cline{1-2}
\endlastfoot\raggedright \_\-\_\-a\-b\-s\-t\-r\-a\-c\-t\-m\-e\-t\-h\-o\-d\-s\-\_\-\_\- & \raggedright \textbf{Value:} 
{\tt \texttt{frozenset([}\texttt{'}\texttt{lines}\texttt{'}\texttt{])}}&\\
\cline{1-2}
\multicolumn{2}{|l|}{\textit{Inherited from theia.optics.optic.Optic \textit{(Section \ref{theia:optics:optic:Optic})}}}\\
\multicolumn{2}{|p{\varwidth}|}{\raggedright Name, OptCount}\\
\cline{1-2}
\multicolumn{2}{|l|}{\textit{Inherited from theia.optics.component.SetupComponent \textit{(Section \ref{theia:optics:component:SetupComponent})}}}\\
\multicolumn{2}{|p{\varwidth}|}{\raggedright SetupCount}\\
\cline{1-2}
\end{longtable}

    \index{theia \textit{(package)}!theia.optics \textit{(package)}!theia.optics.lens \textit{(module)}!theia.optics.lens.Lens \textit{(class)}|)}
    \index{theia \textit{(package)}!theia.optics \textit{(package)}!theia.optics.lens \textit{(module)}|)}
