%
% API Documentation for theia
% Module theia.optics.beamsplitter
%
% Generated by epydoc 3.0.1
% [Fri Aug 18 08:24:16 2017]
%

%%%%%%%%%%%%%%%%%%%%%%%%%%%%%%%%%%%%%%%%%%%%%%%%%%%%%%%%%%%%%%%%%%%%%%%%%%%
%%                          Module Description                           %%
%%%%%%%%%%%%%%%%%%%%%%%%%%%%%%%%%%%%%%%%%%%%%%%%%%%%%%%%%%%%%%%%%%%%%%%%%%%

    \index{theia \textit{(package)}!theia.optics \textit{(package)}!theia.optics.beamsplitter \textit{(module)}|(}
\section{Module theia.optics.beamsplitter}

    \label{theia:optics:beamsplitter}
Defines the BeamSplitter class for theia.


%%%%%%%%%%%%%%%%%%%%%%%%%%%%%%%%%%%%%%%%%%%%%%%%%%%%%%%%%%%%%%%%%%%%%%%%%%%
%%                           Class Description                           %%
%%%%%%%%%%%%%%%%%%%%%%%%%%%%%%%%%%%%%%%%%%%%%%%%%%%%%%%%%%%%%%%%%%%%%%%%%%%

    \index{theia \textit{(package)}!theia.optics \textit{(package)}!theia.optics.beamsplitter \textit{(module)}!theia.optics.beamsplitter.BeamSplitter \textit{(class)}|(}
\subsection{Class BeamSplitter}

    \label{theia:optics:beamsplitter:BeamSplitter}
\begin{tabular}{cccccccccc}
% Line for object, linespec=[False, False, False]
\multicolumn{2}{r}{\settowidth{\BCL}{object}\multirow{2}{\BCL}{object}}
&&
&&
&&
  \\\cline{3-3}
  &&\multicolumn{1}{c|}{}
&&
&&
&&
  \\
% Line for theia.optics.component.SetupComponent, linespec=[False, False]
\multicolumn{4}{r}{\settowidth{\BCL}{theia.optics.component.SetupComponent}\multirow{2}{\BCL}{theia.optics.component.SetupComponent}}
&&
&&
  \\\cline{5-5}
  &&&&\multicolumn{1}{c|}{}
&&
&&
  \\
% Line for theia.optics.optic.Optic, linespec=[False]
\multicolumn{6}{r}{\settowidth{\BCL}{theia.optics.optic.Optic}\multirow{2}{\BCL}{theia.optics.optic.Optic}}
&&
  \\\cline{7-7}
  &&&&&&\multicolumn{1}{c|}{}
&&
  \\
&&&&&&\multicolumn{2}{l}{\textbf{theia.optics.beamsplitter.BeamSplitter}}
\end{tabular}

\begin{alltt}


Beamsplitter class.

This class represents beam splitters composed of two faces (HR, AR)
and with a wedge angle. These are the objects with which the beams will
interact during the ray tracing. Please see the documentation for details
on the geometric construction of these optics.

Beam Splitters behave exactly like mirrors, except that:
    * The default values for transmittances and reflectivities are different
    * Beam splitters never increase the order upon interaction of beams.

Actions:
    * T on HR: 0
    * R on HR: 0
    * T on AR: 0
    * R on AR: 0

*=== Additional attributes with respect to the Optic class ===*

None

*=== Name ===*

BeamSplitter

**Note**: the curvature of any surface is positive for a concave surface
(coating inside the sphere).
Thus kurv*HRNorm/{\textbar}kurv{\textbar} always points to the center
of the sphere of the surface, as is the convention for the lineSurfInter of
geometry module. Same for AR.

*******     HRK {\textgreater} 0 and ARK {\textgreater} 0     *******           HRK {\textgreater} 0 and ARK {\textless} 0
 *****                               ********         and {\textbar}ARK{\textbar} {\textgreater} {\textbar}HRK{\textbar}
 H***A                               H*********A
 *****                               ********
*******                             *******
\end{alltt}


%%%%%%%%%%%%%%%%%%%%%%%%%%%%%%%%%%%%%%%%%%%%%%%%%%%%%%%%%%%%%%%%%%%%%%%%%%%
%%                                Methods                                %%
%%%%%%%%%%%%%%%%%%%%%%%%%%%%%%%%%%%%%%%%%%%%%%%%%%%%%%%%%%%%%%%%%%%%%%%%%%%

  \subsubsection{Methods}

    \vspace{0.5ex}

\hspace{.8\funcindent}\begin{boxedminipage}{\funcwidth}

    \raggedright \textbf{\_\_init\_\_}(\textit{self}, \textit{Wedge}={\tt 0.}, \textit{Alpha}={\tt 0.}, \textit{X}={\tt 0.}, \textit{Y}={\tt 0.}, \textit{Z}={\tt 0.}, \textit{Theta}={\tt pi/2.}, \textit{Phi}={\tt 0.}, \textit{Diameter}={\tt 10.e-2}, \textit{HRr}={\tt .5}, \textit{HRt}={\tt .5}, \textit{ARr}={\tt .1}, \textit{ARt}={\tt .9}, \textit{HRK}={\tt 0.}, \textit{ARK}={\tt 0}, \textit{Thickness}={\tt 2.e-2}, \textit{N}={\tt 1.4585}, \textit{KeepI}={\tt False}, \textit{Ref}={\tt None})

    \vspace{-1.5ex}

    \rule{\textwidth}{0.5\fboxrule}
\setlength{\parskip}{2ex}
    BeamSplitter initializer.

    Parameters are the attributes.

    Returns a beam splitter.

\setlength{\parskip}{1ex}
      Overrides: object.\_\_init\_\_

    \end{boxedminipage}

    \vspace{0.5ex}

\hspace{.8\funcindent}\begin{boxedminipage}{\funcwidth}

    \raggedright \textbf{lines}(\textit{self})

    \vspace{-1.5ex}

    \rule{\textwidth}{0.5\fboxrule}
\setlength{\parskip}{2ex}
    Returns the list of lines necessary to print the object.

\setlength{\parskip}{1ex}
      Overrides: theia.optics.component.SetupComponent.lines

    \end{boxedminipage}


\large{\textbf{\textit{Inherited from theia.optics.optic.Optic\textit{(Section \ref{theia:optics:optic:Optic})}}}}

\begin{quote}
apexes(), collision(), geoCheck(), hit(), hitAR(), hitHR(), hitSide(), isHit(), isHitDics(), translate()
\end{quote}

\large{\textbf{\textit{Inherited from theia.optics.component.SetupComponent\textit{(Section \ref{theia:optics:component:SetupComponent})}}}}

\begin{quote}
\_\_str\_\_()
\end{quote}

\large{\textbf{\textit{Inherited from object}}}

\begin{quote}
\_\_delattr\_\_(), \_\_format\_\_(), \_\_getattribute\_\_(), \_\_hash\_\_(), \_\_new\_\_(), \_\_reduce\_\_(), \_\_reduce\_ex\_\_(), \_\_repr\_\_(), \_\_setattr\_\_(), \_\_sizeof\_\_(), \_\_subclasshook\_\_()
\end{quote}

%%%%%%%%%%%%%%%%%%%%%%%%%%%%%%%%%%%%%%%%%%%%%%%%%%%%%%%%%%%%%%%%%%%%%%%%%%%
%%                              Properties                               %%
%%%%%%%%%%%%%%%%%%%%%%%%%%%%%%%%%%%%%%%%%%%%%%%%%%%%%%%%%%%%%%%%%%%%%%%%%%%

  \subsubsection{Properties}

    \vspace{-1cm}
\hspace{\varindent}\begin{longtable}{|p{\varnamewidth}|p{\vardescrwidth}|l}
\cline{1-2}
\cline{1-2} \centering \textbf{Name} & \centering \textbf{Description}& \\
\cline{1-2}
\endhead\cline{1-2}\multicolumn{3}{r}{\small\textit{continued on next page}}\\\endfoot\cline{1-2}
\endlastfoot\multicolumn{2}{|l|}{\textit{Inherited from object}}\\
\multicolumn{2}{|p{\varwidth}|}{\raggedright \_\_class\_\_}\\
\cline{1-2}
\end{longtable}


%%%%%%%%%%%%%%%%%%%%%%%%%%%%%%%%%%%%%%%%%%%%%%%%%%%%%%%%%%%%%%%%%%%%%%%%%%%
%%                            Class Variables                            %%
%%%%%%%%%%%%%%%%%%%%%%%%%%%%%%%%%%%%%%%%%%%%%%%%%%%%%%%%%%%%%%%%%%%%%%%%%%%

  \subsubsection{Class Variables}

    \vspace{-1cm}
\hspace{\varindent}\begin{longtable}{|p{\varnamewidth}|p{\vardescrwidth}|l}
\cline{1-2}
\cline{1-2} \centering \textbf{Name} & \centering \textbf{Description}& \\
\cline{1-2}
\endhead\cline{1-2}\multicolumn{3}{r}{\small\textit{continued on next page}}\\\endfoot\cline{1-2}
\endlastfoot\raggedright N\-a\-m\-e\- & \raggedright \textbf{Value:} 
{\tt "BeamSplitter"}&\\
\cline{1-2}
\multicolumn{2}{|l|}{\textit{Inherited from theia.optics.optic.Optic \textit{(Section \ref{theia:optics:optic:Optic})}}}\\
\multicolumn{2}{|p{\varwidth}|}{\raggedright OptCount}\\
\cline{1-2}
\multicolumn{2}{|l|}{\textit{Inherited from theia.optics.component.SetupComponent \textit{(Section \ref{theia:optics:component:SetupComponent})}}}\\
\multicolumn{2}{|p{\varwidth}|}{\raggedright SetupCount, \_\_abstractmethods\_\_}\\
\cline{1-2}
\end{longtable}

    \index{theia \textit{(package)}!theia.optics \textit{(package)}!theia.optics.beamsplitter \textit{(module)}!theia.optics.beamsplitter.BeamSplitter \textit{(class)}|)}
    \index{theia \textit{(package)}!theia.optics \textit{(package)}!theia.optics.beamsplitter \textit{(module)}|)}
