%
% API Documentation for theia
% Module theia.optics.beam
%
% Generated by epydoc 3.0.1
% [Thu Aug 24 10:17:28 2017]
%

%%%%%%%%%%%%%%%%%%%%%%%%%%%%%%%%%%%%%%%%%%%%%%%%%%%%%%%%%%%%%%%%%%%%%%%%%%%
%%                          Module Description                           %%
%%%%%%%%%%%%%%%%%%%%%%%%%%%%%%%%%%%%%%%%%%%%%%%%%%%%%%%%%%%%%%%%%%%%%%%%%%%

    \index{theia \textit{(package)}!theia.optics \textit{(package)}!theia.optics.beam \textit{(module)}|(}
\section{Module theia.optics.beam}

    \label{theia:optics:beam}
Defines the GaussianBeam class for theia.


%%%%%%%%%%%%%%%%%%%%%%%%%%%%%%%%%%%%%%%%%%%%%%%%%%%%%%%%%%%%%%%%%%%%%%%%%%%
%%                               Functions                               %%
%%%%%%%%%%%%%%%%%%%%%%%%%%%%%%%%%%%%%%%%%%%%%%%%%%%%%%%%%%%%%%%%%%%%%%%%%%%

  \subsection{Functions}

    \label{theia:optics:beam:userGaussianBeam}
    \index{theia \textit{(package)}!theia.optics \textit{(package)}!theia.optics.beam \textit{(module)}!theia.optics.beam.userGaussianBeam \textit{(function)}}

    \vspace{0.5ex}

\hspace{.8\funcindent}\begin{boxedminipage}{\funcwidth}

    \raggedright \textbf{userGaussianBeam}(\textit{Wx}={\tt 0.001}, \textit{Wy}={\tt 0.001}, \textit{WDistx}={\tt 0.0}, \textit{WDisty}={\tt 0.0}, \textit{Wl}={\tt 1.064e-06}, \textit{P}={\tt 1.0}, \textit{X}={\tt 0.0}, \textit{Y}={\tt 0.0}, \textit{Z}={\tt 0.0}, \textit{Theta}={\tt 1.57079632679}, \textit{Phi}={\tt 0.0}, \textit{Alpha}={\tt 0.0}, \textit{Ref}={\tt None})

    \vspace{-1.5ex}

    \rule{\textwidth}{0.5\fboxrule}
\setlength{\parskip}{2ex}
    Constructor used for user inputted beams, separated from the class 
    initializer because the internal state of a beam is very different from
    the input of this user-defined beam.

    Input parameters are processed to make arguments for the class 
    constructor and then the corresponding beam is returned.

\setlength{\parskip}{1ex}
    \end{boxedminipage}


%%%%%%%%%%%%%%%%%%%%%%%%%%%%%%%%%%%%%%%%%%%%%%%%%%%%%%%%%%%%%%%%%%%%%%%%%%%
%%                               Variables                               %%
%%%%%%%%%%%%%%%%%%%%%%%%%%%%%%%%%%%%%%%%%%%%%%%%%%%%%%%%%%%%%%%%%%%%%%%%%%%

  \subsection{Variables}

    \vspace{-1cm}
\hspace{\varindent}\begin{longtable}{|p{\varnamewidth}|p{\vardescrwidth}|l}
\cline{1-2}
\cline{1-2} \centering \textbf{Name} & \centering \textbf{Description}& \\
\cline{1-2}
\endhead\cline{1-2}\multicolumn{3}{r}{\small\textit{continued on next page}}\\\endfoot\cline{1-2}
\endlastfoot\raggedright \_\-\_\-p\-a\-c\-k\-a\-g\-e\-\_\-\_\- & \raggedright \textbf{Value:} 
{\tt \texttt{'}\texttt{theia.optics}\texttt{'}}&\\
\cline{1-2}
\end{longtable}


%%%%%%%%%%%%%%%%%%%%%%%%%%%%%%%%%%%%%%%%%%%%%%%%%%%%%%%%%%%%%%%%%%%%%%%%%%%
%%                           Class Description                           %%
%%%%%%%%%%%%%%%%%%%%%%%%%%%%%%%%%%%%%%%%%%%%%%%%%%%%%%%%%%%%%%%%%%%%%%%%%%%

    \index{theia \textit{(package)}!theia.optics \textit{(package)}!theia.optics.beam \textit{(module)}!theia.optics.beam.GaussianBeam \textit{(class)}|(}
\subsection{Class GaussianBeam}

    \label{theia:optics:beam:GaussianBeam}
\begin{tabular}{cccccc}
% Line for object, linespec=[False]
\multicolumn{2}{r}{\settowidth{\BCL}{object}\multirow{2}{\BCL}{object}}
&&
  \\\cline{3-3}
  &&\multicolumn{1}{c|}{}
&&
  \\
&&\multicolumn{2}{l}{\textbf{theia.optics.beam.GaussianBeam}}
\end{tabular}

\begin{alltt}


GaussianBeam class.

This class represents general astigmatic Gaussian beams in 3D space.
These are the objects that are intended to interact with the optical
components during the ray tracing and that are rendered in 3D thanks to
FreeCAD.

*=== Attributes ===*
BeamCount: class attribute, counts beams. [integer]
Name: class attribute. [string]
QTens: general astigmatic complex curvature tensor at the origin.
    [np. array of complex]
N: Refraction index of the medium in which the beam is placed. [float]
Wl: Wave-length in vacuum of the beam (frequency never changes). [float]
P: Power of the beam. [float]
Pos: Position in 3D space of the origin of the beam. [3D vector]
Dir: Normalized direction in 3D space of the beam axis. [3D vector]
U: A tuple of unitary vectors which along with Dir form a direct orthonormal
    basis in which the Q tensor is expressed. [tuple of 3D vectors]
Ref: Reference to the beam. [string]
OptDist: Optical length. [float]
Length: Geometrical length of the beam. [float]
StrayOrder: Number representing the *strayness* of the beam. If the beams
    results from a transmission on a HR surface or a reflection on a AR
    surface, then its StrayOrder is the StrayOrder of the parent beam + 1.
    [integer]
Optic: Ref of optic where the beam departs from ('Laser' if laser). [string]
Face: Face of the optic where the beam departs from. [string]
TargetOptic: Ref of the optic where the beam terminates (None if open
    beam). [string]
TargetFace: Face of the target optic where the beam terminates. [string]
DWx: Distance of waist on X. [float]
DWy: Distance of waist on Y. [float]
Wx: Waist on X. [float]
Wy: Waist on Y. [float]
IWx: Width of beam on X at origin. [float]
IWy: Width of beam on Y at origin. [float]
TWx: Width of beam on X at target surface (None if open beam). [float]
TWy: Width of beam on Y at target surface (None if open beam).
\end{alltt}


%%%%%%%%%%%%%%%%%%%%%%%%%%%%%%%%%%%%%%%%%%%%%%%%%%%%%%%%%%%%%%%%%%%%%%%%%%%
%%                                Methods                                %%
%%%%%%%%%%%%%%%%%%%%%%%%%%%%%%%%%%%%%%%%%%%%%%%%%%%%%%%%%%%%%%%%%%%%%%%%%%%

  \subsubsection{Methods}

    \vspace{0.5ex}

\hspace{.8\funcindent}\begin{boxedminipage}{\funcwidth}

    \raggedright \textbf{\_\_init\_\_}(\textit{self}, \textit{Q}, \textit{N}, \textit{Wl}, \textit{P}, \textit{Pos}, \textit{Dir}, \textit{Ux}, \textit{Uy}, \textit{Ref}, \textit{OptDist}, \textit{Length}, \textit{StrayOrder}, \textit{Optic}, \textit{Face})

    \vspace{-1.5ex}

    \rule{\textwidth}{0.5\fboxrule}
\setlength{\parskip}{2ex}
    Beam initializer.

    This is the initializer used internally for beam creation, for user 
    inputted beams, see function userGaussianBeam.

    Returns a Gaussian beam with attributes as the parameters.

\setlength{\parskip}{1ex}
      Overrides: object.\_\_init\_\_

    \end{boxedminipage}

    \vspace{0.5ex}

\hspace{.8\funcindent}\begin{boxedminipage}{\funcwidth}

    \raggedright \textbf{\_\_str\_\_}(\textit{self})

    \vspace{-1.5ex}

    \rule{\textwidth}{0.5\fboxrule}
\setlength{\parskip}{2ex}
    String representation of the beam, when calling print(beam).

\setlength{\parskip}{1ex}
      Overrides: object.\_\_str\_\_

    \end{boxedminipage}

    \label{theia:optics:beam:GaussianBeam:lines}
    \index{theia \textit{(package)}!theia.optics \textit{(package)}!theia.optics.beam \textit{(module)}!theia.optics.beam.GaussianBeam \textit{(class)}!theia.optics.beam.GaussianBeam.lines \textit{(method)}}

    \vspace{0.5ex}

\hspace{.8\funcindent}\begin{boxedminipage}{\funcwidth}

    \raggedright \textbf{lines}(\textit{self})

    \vspace{-1.5ex}

    \rule{\textwidth}{0.5\fboxrule}
\setlength{\parskip}{2ex}
    Returns the list of lines necessary to print the object.

\setlength{\parskip}{1ex}
    \end{boxedminipage}

    \label{theia:optics:beam:GaussianBeam:Q}
    \index{theia \textit{(package)}!theia.optics \textit{(package)}!theia.optics.beam \textit{(module)}!theia.optics.beam.GaussianBeam \textit{(class)}!theia.optics.beam.GaussianBeam.Q \textit{(method)}}

    \vspace{0.5ex}

\hspace{.8\funcindent}\begin{boxedminipage}{\funcwidth}

    \raggedright \textbf{Q}(\textit{self}, \textit{d}={\tt 0.0})

    \vspace{-1.5ex}

    \rule{\textwidth}{0.5\fboxrule}
\setlength{\parskip}{2ex}
    Return the Q tensor at a distance d of origin.

\setlength{\parskip}{1ex}
    \end{boxedminipage}

    \label{theia:optics:beam:GaussianBeam:QParam}
    \index{theia \textit{(package)}!theia.optics \textit{(package)}!theia.optics.beam \textit{(module)}!theia.optics.beam.GaussianBeam \textit{(class)}!theia.optics.beam.GaussianBeam.QParam \textit{(method)}}

    \vspace{0.5ex}

\hspace{.8\funcindent}\begin{boxedminipage}{\funcwidth}

    \raggedright \textbf{QParam}(\textit{self}, \textit{d}={\tt 0.0})

    \vspace{-1.5ex}

    \rule{\textwidth}{0.5\fboxrule}
\setlength{\parskip}{2ex}
\begin{alltt}
Compute the complex parameters q1 and q2 and theta of beam.

What is implemented here is a straightforward calculation to extract
the q1, q2, and theta of the normal form of Q.

    Returns a tuple q1, q2, theta
\end{alltt}

\setlength{\parskip}{1ex}
    \end{boxedminipage}

    \label{theia:optics:beam:GaussianBeam:ROC}
    \index{theia \textit{(package)}!theia.optics \textit{(package)}!theia.optics.beam \textit{(module)}!theia.optics.beam.GaussianBeam \textit{(class)}!theia.optics.beam.GaussianBeam.ROC \textit{(method)}}

    \vspace{0.5ex}

\hspace{.8\funcindent}\begin{boxedminipage}{\funcwidth}

    \raggedright \textbf{ROC}(\textit{self}, \textit{dist}={\tt 0.0})

    \vspace{-1.5ex}

    \rule{\textwidth}{0.5\fboxrule}
\setlength{\parskip}{2ex}
    Return the tuple of ROC of the beam.

\setlength{\parskip}{1ex}
    \end{boxedminipage}

    \label{theia:optics:beam:GaussianBeam:waistPos}
    \index{theia \textit{(package)}!theia.optics \textit{(package)}!theia.optics.beam \textit{(module)}!theia.optics.beam.GaussianBeam \textit{(class)}!theia.optics.beam.GaussianBeam.waistPos \textit{(method)}}

    \vspace{0.5ex}

\hspace{.8\funcindent}\begin{boxedminipage}{\funcwidth}

    \raggedright \textbf{waistPos}(\textit{self})

    \vspace{-1.5ex}

    \rule{\textwidth}{0.5\fboxrule}
\setlength{\parskip}{2ex}
    Return the tuple of positions of the waists of the beam along Dir.

\setlength{\parskip}{1ex}
    \end{boxedminipage}

    \label{theia:optics:beam:GaussianBeam:rayleigh}
    \index{theia \textit{(package)}!theia.optics \textit{(package)}!theia.optics.beam \textit{(module)}!theia.optics.beam.GaussianBeam \textit{(class)}!theia.optics.beam.GaussianBeam.rayleigh \textit{(method)}}

    \vspace{0.5ex}

\hspace{.8\funcindent}\begin{boxedminipage}{\funcwidth}

    \raggedright \textbf{rayleigh}(\textit{self})

    \vspace{-1.5ex}

    \rule{\textwidth}{0.5\fboxrule}
\setlength{\parskip}{2ex}
    Return the tuple of Rayleigh ranges of the beam.

\setlength{\parskip}{1ex}
    \end{boxedminipage}

    \label{theia:optics:beam:GaussianBeam:width}
    \index{theia \textit{(package)}!theia.optics \textit{(package)}!theia.optics.beam \textit{(module)}!theia.optics.beam.GaussianBeam \textit{(class)}!theia.optics.beam.GaussianBeam.width \textit{(method)}}

    \vspace{0.5ex}

\hspace{.8\funcindent}\begin{boxedminipage}{\funcwidth}

    \raggedright \textbf{width}(\textit{self}, \textit{d}={\tt 0.0})

    \vspace{-1.5ex}

    \rule{\textwidth}{0.5\fboxrule}
\setlength{\parskip}{2ex}
    Return the tuple of beam widths at distance d.

\setlength{\parskip}{1ex}
    \end{boxedminipage}

    \label{theia:optics:beam:GaussianBeam:waistSize}
    \index{theia \textit{(package)}!theia.optics \textit{(package)}!theia.optics.beam \textit{(module)}!theia.optics.beam.GaussianBeam \textit{(class)}!theia.optics.beam.GaussianBeam.waistSize \textit{(method)}}

    \vspace{0.5ex}

\hspace{.8\funcindent}\begin{boxedminipage}{\funcwidth}

    \raggedright \textbf{waistSize}(\textit{self})

    \vspace{-1.5ex}

    \rule{\textwidth}{0.5\fboxrule}
\setlength{\parskip}{2ex}
    Return a tuple with the waist sizes in x and y.

\setlength{\parskip}{1ex}
    \end{boxedminipage}

    \label{theia:optics:beam:GaussianBeam:gouy}
    \index{theia \textit{(package)}!theia.optics \textit{(package)}!theia.optics.beam \textit{(module)}!theia.optics.beam.GaussianBeam \textit{(class)}!theia.optics.beam.GaussianBeam.gouy \textit{(method)}}

    \vspace{0.5ex}

\hspace{.8\funcindent}\begin{boxedminipage}{\funcwidth}

    \raggedright \textbf{gouy}(\textit{self}, \textit{d}={\tt 0.0})

    \vspace{-1.5ex}

    \rule{\textwidth}{0.5\fboxrule}
\setlength{\parskip}{2ex}
    Return the tuple of Gouy phases.

\setlength{\parskip}{1ex}
    \end{boxedminipage}

    \label{theia:optics:beam:GaussianBeam:initGaussianData}
    \index{theia \textit{(package)}!theia.optics \textit{(package)}!theia.optics.beam \textit{(module)}!theia.optics.beam.GaussianBeam \textit{(class)}!theia.optics.beam.GaussianBeam.initGaussianData \textit{(method)}}

    \vspace{0.5ex}

\hspace{.8\funcindent}\begin{boxedminipage}{\funcwidth}

    \raggedright \textbf{initGaussianData}(\textit{self})

    \vspace{-1.5ex}

    \rule{\textwidth}{0.5\fboxrule}
\setlength{\parskip}{2ex}
    Writes the relevant DW, W, IW data with Q.

    Is called upon construction to write the data of waist position and 
    size, initial widths once and for all.

\setlength{\parskip}{1ex}
    \end{boxedminipage}

    \label{theia:optics:beam:GaussianBeam:translate}
    \index{theia \textit{(package)}!theia.optics \textit{(package)}!theia.optics.beam \textit{(module)}!theia.optics.beam.GaussianBeam \textit{(class)}!theia.optics.beam.GaussianBeam.translate \textit{(method)}}

    \vspace{0.5ex}

\hspace{.8\funcindent}\begin{boxedminipage}{\funcwidth}

    \raggedright \textbf{translate}(\textit{self}, \textit{X}={\tt 0.0}, \textit{Y}={\tt 0.0}, \textit{Z}={\tt 0.0})

    \vspace{-1.5ex}

    \rule{\textwidth}{0.5\fboxrule}
\setlength{\parskip}{2ex}
    Move the beam to (current position + (X, Y, Z)).

    X, Y, Z: components of the translation vector.

    No return value.

\setlength{\parskip}{1ex}
    \end{boxedminipage}


\large{\textbf{\textit{Inherited from object}}}

\begin{quote}
\_\_delattr\_\_(), \_\_format\_\_(), \_\_getattribute\_\_(), \_\_hash\_\_(), \_\_new\_\_(), \_\_reduce\_\_(), \_\_reduce\_ex\_\_(), \_\_repr\_\_(), \_\_setattr\_\_(), \_\_sizeof\_\_(), \_\_subclasshook\_\_()
\end{quote}

%%%%%%%%%%%%%%%%%%%%%%%%%%%%%%%%%%%%%%%%%%%%%%%%%%%%%%%%%%%%%%%%%%%%%%%%%%%
%%                              Properties                               %%
%%%%%%%%%%%%%%%%%%%%%%%%%%%%%%%%%%%%%%%%%%%%%%%%%%%%%%%%%%%%%%%%%%%%%%%%%%%

  \subsubsection{Properties}

    \vspace{-1cm}
\hspace{\varindent}\begin{longtable}{|p{\varnamewidth}|p{\vardescrwidth}|l}
\cline{1-2}
\cline{1-2} \centering \textbf{Name} & \centering \textbf{Description}& \\
\cline{1-2}
\endhead\cline{1-2}\multicolumn{3}{r}{\small\textit{continued on next page}}\\\endfoot\cline{1-2}
\endlastfoot\multicolumn{2}{|l|}{\textit{Inherited from object}}\\
\multicolumn{2}{|p{\varwidth}|}{\raggedright \_\_class\_\_}\\
\cline{1-2}
\end{longtable}


%%%%%%%%%%%%%%%%%%%%%%%%%%%%%%%%%%%%%%%%%%%%%%%%%%%%%%%%%%%%%%%%%%%%%%%%%%%
%%                            Class Variables                            %%
%%%%%%%%%%%%%%%%%%%%%%%%%%%%%%%%%%%%%%%%%%%%%%%%%%%%%%%%%%%%%%%%%%%%%%%%%%%

  \subsubsection{Class Variables}

    \vspace{-1cm}
\hspace{\varindent}\begin{longtable}{|p{\varnamewidth}|p{\vardescrwidth}|l}
\cline{1-2}
\cline{1-2} \centering \textbf{Name} & \centering \textbf{Description}& \\
\cline{1-2}
\endhead\cline{1-2}\multicolumn{3}{r}{\small\textit{continued on next page}}\\\endfoot\cline{1-2}
\endlastfoot\raggedright B\-e\-a\-m\-C\-o\-u\-n\-t\- & \raggedright \textbf{Value:} 
{\tt 0}&\\
\cline{1-2}
\raggedright N\-a\-m\-e\- & \raggedright \textbf{Value:} 
{\tt \texttt{'}\texttt{Beam}\texttt{'}}&\\
\cline{1-2}
\end{longtable}

    \index{theia \textit{(package)}!theia.optics \textit{(package)}!theia.optics.beam \textit{(module)}!theia.optics.beam.GaussianBeam \textit{(class)}|)}
    \index{theia \textit{(package)}!theia.optics \textit{(package)}!theia.optics.beam \textit{(module)}|)}
